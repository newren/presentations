\ifx\handout\undefined
  \documentclass[compress,t]{beamer}
\else
  \documentclass[compress,handout,t]{beamer}
  \usepackage{pgfpages}
  %\pgfpagesuselayout{8 on 1}[letterpaper,border shrink=1mm]
  \pgfpagesuselayout{8 on 1}[letterpaper,border shrink=0mm]
\fi

\usepackage{amsmath}                  % Nice math stuff
\usepackage{amsfonts}                 % Extra math fonts (i.e., \mathbb{})
\usepackage{amssymb}                  % Extra meth symbols
\usepackage[mathscr]{eucal}           % Scripted math fonts
\usepackage{listings}                 % Formatting code listings
\usepackage{verbatim}
%\usepackage{picins}
\usepackage{colortbl}
\usepackage{fancybox}
\usepackage{textpos}
\usepackage{color}

\mode<presentation>
{
  \usetheme{Warsaw} % or ...
  \setbeamercovered{transparent} % or whatever (possibly just delete it)
  \usefonttheme[onlymath]{serif}
}
\mode<beamer>
{
  \AtBeginSection[]{
    \begin{frame}
      \frametitle{Outline}
      \begin{textblock}{0.9}(0.1,0.0)
        {\small\tableofcontents[currentsection,hideothersubsections]}
      \end{textblock}
    \end{frame}
  }
}
% PKN: the footline isn't very useful -- better to conserve pixels
\setbeamertemplate{footline}{}
\setbeamertemplate{navigation symbols}{}
\setbeamersize{text margin left=0.2cm}
\setbeamersize{text margin right=0.2cm}
\setbeamersize{sidebar width left=0cm}
\setbeamersize{sidebar width right=0cm}
%\setbeamertemplate{footline}
%{% 
%\begin{beamercolorbox}{section in head/foot} 
%\vskip2pt\insertnavigation{\paperwidth}\vskip2pt 
%\vskip2pt
%\hfill Page \insertpagenumber \hspace{1em}
%\end{beamercolorbox}% 
%} 

\usepackage[english]{babel}           % or whatever
\usepackage[latin1]{inputenc}         % or whatever
\usepackage{times}
\usepackage[T1]{fontenc}

\newcommand{\Comment}[1]{{\usebeamercolor[fg]{item}#1}}
\newcommand{\Text}[1]{\texttt{#1}}
\xdefinecolor{TokenColor}{rgb}{0.000, 0.700, 0.000}
\newcommand{\Token}[1]{{\color{TokenColor}#1}}
\xdefinecolor{HiddenColor}{rgb}{0.800, 0.800, 0.800}
\newcommand{\Hidden}[1]{{\color{HiddenColor}#1}}
\xdefinecolor{MyRed}{rgb}{0.800, 0.200, 0.200}
\newcommand{\Red}[1]{{\color{MyRed}#1}}
\xdefinecolor{MyWhite}{rgb}{1.0, 1.0, 1.0}
\xdefinecolor{MyGrey}{rgb}{0.900, 0.900, 0.900}

\setlength{\TPHorizModule}{\textwidth}
\setlength{\TPVertModule}{\textheight}

\lstloadlanguages{C++}
\lstset{%
  basicstyle=\ttfamily\scriptsize\bfseries, %
  commentstyle=\usebeamercolor[fg]{item}, %
  stringstyle=\color{TokenColor}, %
  language=C++}

%% Typeset an example command line input.
%\newcommand{\cmd}[1]{\textbf{\texttt{#1}}}
%\newcommand{\cl}[1]{\$ \textbf{#1}}
%\newcommand{\cls}[1]{\texttt{\scriptsize\$ \textbf{#1}}}
%\newcommand{\clsnp}[1]{\texttt{\scriptsize\textbf{#1}}}
\newcommand{\hilight}{\color{blue}}
\xdefinecolor{MyGreen}{rgb}{0.000, 0.500, 0.000} % green
\newcommand{\nonliteral}[1]{{\color{MyGreen}#1}}
\newcommand{\cmd}[1]{\texttt{\hilight#1}}
\newcommand{\cl}[1]{\$ \cmd{#1}}

%% typset a URL in a uniform way
\newcommand{\Link}[1]{{\usebeamercolor[fg]{item}\underline{\url{#1}}}}

\usepackage{multicol}
\usepackage{ulem} % used in definition of \redout

\usepackage{tikz}
\usetikzlibrary{arrows,snakes,backgrounds,shapes}
\tikzstyle{commit}=[circle,draw=blue!50,fill=blue!20]
\tikzstyle{redCommit}=[circle,draw=MyRed!50,fill=MyRed!20]
\tikzstyle{orphanCommit}=[black!50,circle,draw=blue!20,fill=blue!10]
\tikzstyle{droppedCommit}=[black!60,circle,draw=MyRed!50,fill=MyRed!20,double]
\tikzstyle{ref}=[rectangle,draw=black!50,fill=black!20]
\tikzstyle{specialRef}=[black!50,rectangle,draw=black!50,fill=black!10]
\tikzstyle{msg}=[tape,tape bend top=none,draw=MyRed!50,fill=MyRed!20]
\tikzstyle{prnt}=[thick,-stealth']
\tikzstyle{oprnt}=[black!50,dotted,-stealth',draw=black!50]
\tikzstyle{bptrnext}=[black!80,dashed,-stealth',draw=black!80]
\tikzstyle{bptr}=[thick,-latex']
\tikzstyle{nextmsg}=[thick,-open triangle 45]
\newcommand{\parent}[1]{\char94{#1}}
\newcommand{\ancestor}[1]{\char126{#1}}
\colorlet{myred}{red!90!black} % slightly darker than normal red
\colorlet{myblue}{cyan}        % plain cyan works; might consdier darkening
\colorlet{mygreen}{green!70!black} % slightly darker than normal green
\colorlet{mypurple}{red!45!blue}
\definecolor{mybrown}{rgb}{0.45,0.28,0.13}
\newcommand\redout{\bgroup\markoverwith
    {\textcolor{myred}{\rule[.5ex]{2pt}{0.8pt}}}\ULon
}

\title{Scaling the Merge Machinery}

\author{Elijah Newren}
\institute{Palantir Technologies}

%\date{January 8, 2014}
%\date{Palantir Technologies}
\date{}

%\includeonlyframes{wip}

\begin{document}

%%%%%%%%%%%%%%%%%%%%%%%%%%%%%%%%%%%%%%%%%%%%%%%%%%%%%%%%%%%%%%%%%%%%%%%%%%

\begin{frame}
  \titlepage
  %\mode<handout>{
  %\vfill
  %\begin{center}\scriptsize
  %  \Link{http://wiki.yojoe.local/ITOOLS/git-tutorial}
  %  \vfill
  %  \cmd{git clone gerrit:itools/public/git-tutorial}
  %\end{center}
  %{\tiny\Hidden{\input{VERSION-FILE}}}
  %}
\end{frame}

\ifx\handout\undefined
\title{Git Merge}
\begin{frame}
  \titlepage
\end{frame}

\title{(Git Merge)$^2$}
\begin{frame}
  \titlepage
\end{frame}
\fi

%%%%%%%%%%%%%%%%%%%%%%%%%%%%%%%%%%%%%%%%%%%%%%%%%%%%%%%%%%%%%%%%%%%%%%%%%%
\section[Overhaul]{``Merge machinery''}
%%%%%%%%%%%%%%%%%%%%%%%%%%%%%%%%%%%%%%%%%%%%%%%%%%%%%%%%%%%%%%%%%%%%%%%%%%

\begin{comment}
\subsection{Division}
\begin{frame}
  \frametitle{Division}

  \begin{itemize}[<+->]

    \item xdiff/xmerge.c: 3-way content merging of a single file
    \item ll-merge.c: thin xdl\_merge wrapper; heeds path-specific
          .gitattributes such as special drivers or normalization

    \item merge-recursive.c (default strategy and main engine): finds
          and pre-merges merge-bases, handles path based-conflicts,
          mode-based conflicts (e.g. symlink vs. submodule),
          renames, and wires everything together.
      % merge-recursive is about 2x the size of the other three pieces combined

    \item builtin/merge.c: which strategy (octopus, fast-forward,
          default), diffstat, etc.
  \end{itemize}

\end{frame}
\end{comment}

%%%%%%%%%%%%%%%%%%%%%%%%%%%%%%%%%%%%%%%%%%%%%%%%%%%%%%%%%%%%%%%%%%%%%%%%%%

\subsection{Affected commands}
\begin{frame}
  \frametitle{Affected Commands}

  The merge machinery (merge-recursive) powers several aspects of git:

  \begin{itemize}
    \item merge
    \item cherry-pick
    \item revert
    \item rebase
    \item am -3
    \item stash
    \item checkout -m
  \end{itemize}

\end{frame}

%%%%%%%%%%%%%%%%%%%%%%%%%%%%%%%%%%%%%%%%%%%%%%%%%%%%%%%%%%%%%%%%%%%%%%%%%%

\subsection[Issues]{The journey}
\begin{frame}
  \frametitle{The journey}

  A few years ago...

  \begin{minipage}{0.88\textwidth}
  \begin{block}{Issues starting the journey}
    {\footnotesize
    \begin{itemize}
      \item cherry-pick would fail to detect renames and fail to notify
            about needed merge.renameLimit
      \item cherry-pick would ignore merge.renameLimit > 32767
      \item if directory renames involved, files would be left in wrong
            directory
      \item people wrote custom purpose scripts to cherry-pick things
      \item after fixing merge.renameLimit, cherry-picking small patches
            would take more than 9 minutes.
    \end{itemize}
    }
  \end{block}
  \end{minipage}

\end{frame}

%%%%%%%%%%%%%%%%%%%%%%%%%%%%%%%%%%%%%%%%%%%%%%%%%%%%%%%%%%%%%%%%%%%%%%%%%%

\subsection[Challenges]{Overhaul}
\begin{frame}
  \frametitle{Quotes}

  The two most prolific authors of git opining on merge-recursive:

  \only<2->{
  \begin{itemize}

    \item ``[It is] some pretty hairy code.  Every time I
            start to look at it I get confused and can't remember what
            breakthrough I thought I was close to making before.''
            %
            \href{https://public-inbox.org/git/20110616210551.GA15505@sigill.intra.peff.net/}{(Jeff King)}

    \only<3>{
    \item ``I've written off that code as mostly unsalvageable long
            time ago.''
          \href{https://public-inbox.org/git/xmqqk1ydkbx0.fsf@gitster.mtv.corp.google.com/}{(Junio Hamano)}
    }

  \end{itemize}
  }

\end{frame}

%%%%%%%%%%%%%%%%%%%%%%%%%%%%%%%%%%%%%%%%%%%%%%%%%%%%%%%%%%%%%%%%%%%%%%%%%%

\begin{frame}
  \frametitle{Goals}

  Goals for my rewrite of the machinery are to improve each of:
  \begin{itemize}
    \item Maintainability \& understandability
    \item API Quality (enable new features?)
    \item Correctness
    \only<1| handout:0>{
    \item Performance
    }
    \only<2>{
    \item {\textbf{Performance}}
    }
  \end{itemize}

\end{frame}

%%%%%%%%%%%%%%%%%%%%%%%%%%%%%%%%%%%%%%%%%%%%%%%%%%%%%%%%%%%%%%%%%%%%%%%%%%

\subsection[Strategy types]{Types of performance strategies}
\begin{frame}
  \frametitle{Types of performance strategies}

  \only<1-6| handout:1>{
  I have always enjoyed performance talks; they make me feel smarter:
  \only<2->{
  \begin{itemize}
    \item Squeezing performance out of the hardware
    \only<3->{
    \item Applying ideas from other problem domains to new areas
    \only<4->{
    \item Using clever approximation algorithms to get near solutions
    \only<5->{
    \item Inventing new algorithms
    }}}
  \end{itemize}
  \only<6| handout:1>{
    \begin{tikzpicture}[remember picture, overlay]
      \draw[myred,line width=3pt] ( 0,0.5) -- (11,2.7);
      \draw[myred,line width=3pt] (11,0.5) -- ( 0,2.7);
    \end{tikzpicture}
  }
  }}

  \only<7-| handout:2>{
  Actual performance strategies used:
  \only<8->{
  \begin{itemize}
    %\item Avoid well-known slow constructs
    %\only<9->{
    \item Don't do unnecessary work
    \only<9->{
    \item Don't redo work
    \only<10->{
    \item Don't redo unnecessary work
    \only<11->{
    \item Fudge "unnecessary"
    }}}%}
  \end{itemize}
  }}

\end{frame}

%%%%%%%%%%%%%%%%%%%%%%%%%%%%%%%%%%%%%%%%%%%%%%%%%%%%%%%%%%%%%%%%%%%%%%%%%%

%\subsection*{Warning}
\begin{frame}
  \frametitle{Warning}

  \vfill
  \begin{center}
    \begin{minipage}{0.55\textwidth}
    \begin{itemize}
      \item Glossing over lots of details
      \item Simplifications not fully accurate
    \end{itemize}
    \end{minipage}
  \end{center}
\end{frame}

\begin{comment}
%%%%%%%%%%%%%%%%%%%%%%%%%%%%%%%%%%%%%%%%%%%%%%%%%%%%%%%%%%%%%%%%%%%%%%%%%%
%\section[Simple Wins]{Simple performance wins}
%%%%%%%%%%%%%%%%%%%%%%%%%%%%%%%%%%%%%%%%%%%%%%%%%%%%%%%%%%%%%%%%%%%%%%%%%%

%\subsection[Slow Constructs]{Avoiding slow constructs}
\begin{frame}[fragile]
  \frametitle{Avoiding well-known slow constructs}

  From string-list.h:
  {\scriptsize
  \vspace*{-0.5\baselineskip}
  \begin{verbatim}
 * NOTE: It is more efficient to build an unsorted list and sort it
 * afterwards, instead of building a sorted list (`O(n log n)` instead of
 * `O(n^2)`).
  \end{verbatim}
  \vspace*{-\baselineskip}
  }

  \only<2>{
  Oops.
  }
\end{frame}

%%%%%%%%%%%%%%%%%%%%%%%%%%%%%%%%%%%%%%%%%%%%%%%%%%%%%%%%%%%%%%%%%%%%%%%%%%

\begin{frame}[fragile]
  \frametitle{Avoiding well-known slow constructs}

  Comment above \texttt{remove\_marked\_cache\_entries()}:
  {\scriptsize
  \vspace*{-0.5\baselineskip}
  \begin{verbatim}
/*
 * Remove all cache entries marked for removal...this is much more
 * effective than calling remove_index_entry_at() for each entry to be
 * removed.
 */
  \end{verbatim}
  }

  \only<2->{
  \vspace*{-1.25\baselineskip}
  \begin{itemize}
    \item merge-recursive's design incompatible with this, instead:
    \item O($tracked\_files$) work required for each deletion/insertion
    \item O($2*files\_changed$) deletions/insertions
  \end{itemize}
  }

  \only<3->{
    Needed a significant restructure to avoid this quasi-quadratic
    behavior.
  }

\end{frame}

%%%%%%%%%%%%%%%%%%%%%%%%%%%%%%%%%%%%%%%%%%%%%%%%%%%%%%%%%%%%%%%%%%%%%%%%%%

%\subsection[Unnecessary]{Avoiding unnecessary work}
\begin{frame}[fragile]
  \frametitle{Avoiding unnecessary work}

  \begin{center}
  \begin{minipage}{0.83\textwidth}
  \begin{block}{How merge-recursive got its name}
    {\scriptsize
    \begin{semiverbatim}
static struct commit* merge_recursive_internal(side1, side2)
\{
    ...
    \uncover<2->{
    /* multiple merge bases? merge them together first. */\only<1-2| handout:0>{
    combined = bases[0];    }\only<3->{
    {\color{myred}combined = bases[0];}   }
    for (base in bases[1:])
        combined = merge_recursive_internal(combined, base);  \only<1-2|handout:0>{
    discard_index();
    if (!inner_merge)
        repo_read_index();  }\only<3->{{\color{myred}
    discard_index();
    if (!inner_merge)
        repo_read_index();}   }
    }
    result = merge_trees_internal(\only<1|handout:0>{base}\only<2->{combined}, side1, side2);
    ...
\}
    \end{semiverbatim}
    }
  \end{block}
  \end{minipage}
  \end{center}

  %\vspace*{\baselineskip}
  \only<3->{What if there is only one merge base to start? }\only<4>{Oops.}

\end{frame}

%%%%%%%%%%%%%%%%%%%%%%%%%%%%%%%%%%%%%%%%%%%%%%%%%%%%%%%%%%%%%%%%%%%%%%%%%%

%\subsection[Redoing]{Avoiding redoing work}
\begin{frame}
  \frametitle{Avoiding redoing work}

  Tree walking from merge-recursive.c:
  \only<2->{
  \begin{itemize}
    \item unpack\_trees()
    \only<3->{
    \item diff to find renames from base->side1
    \only<4->{
    \item diff to find renames from base->side2
    \only<5->{
    \item get all file and directory names on side1
    \only<6->{
    \item get all file and directory names on side2
    \only<7->{
    \item 6x get stage data \only<8->{for EACH rename \only<9->{on EACH side}}
    }}}}}
  \end{itemize}
  }
  \only<10->{Total: $5+6*number\_of\_renames$}

  \only<11->{
  \vspace*{1.5\baselineskip}
  Tree walking from merge machinery rewrite:
  \only<12->{
  \begin{itemize}
    \item collect\_merge\_info()
    \only<13->{
    \item unpack\_trees()
    }
  \end{itemize}
  }
  \only<14->{Total: $2$}
  }

\end{frame}

%%%%%%%%%%%%%%%%%%%%%%%%%%%%%%%%%%%%%%%%%%%%%%%%%%%%%%%%%%%%%%%%%%%%%%%%%%

\subsection*{But...}
\begin{frame}
  \frametitle{The big kicker...}

  \vfill
  \begin{center}
  \begin{minipage}{0.55\textwidth}
    All of these optimizations combined are often a drop in a bucket
    compared to the cost of rename detection.
  \end{minipage}
  \end{center}

\end{frame}
\end{comment}

%%%%%%%%%%%%%%%%%%%%%%%%%%%%%%%%%%%%%%%%%%%%%%%%%%%%%%%%%%%%%%%%%%%%%%%%%%
\section[Background]{Merging and renames background}
%%%%%%%%%%%%%%%%%%%%%%%%%%%%%%%%%%%%%%%%%%%%%%%%%%%%%%%%%%%%%%%%%%%%%%%%%%

%%%%%%%%%%%%%%%%%%%%%%%%%%%%%%%%%%%%%%%%%%%%%%%%%%%%%%%%%%%%%%%%%%%%%%%%%%
\subsection{Content merge}
%%%%%%%%%%%%%%%%%%%%%%%%%%%%%%%%%%%%%%%%%%%%%%%%%%%%%%%%%%%%%%%%%%%%%%%%%%

\begin{frame}
  \frametitle{Three-way content merge}

  \vspace*{-\baselineskip}
  \begin{multicols}{2}
    File from branch Side1:\\
    {\footnotesize\texttt{%
    %\vspace*{-0.125\baselineskip}%
    \quad{}...                             \\
    \quad{}speak\_like\_a\_pirate(arrrgs); \\
    \quad{}\only<3->{\color{myred}}explore\_sea(aye, matey);\only<3->{\color{black}}\\
    \quad{}shiver(me.timbers);             \\
    %\vspace*{-0.25\baselineskip}%
    \quad{}...                             \\
    }}
    \columnbreak
    \pause
    Same file from branch Side2:\\
    {\footnotesize\texttt{%
    %\vspace*{-0.125\baselineskip}%
    \quad{}...                             \\
    \quad{}speak\_like\_a\_pirate(arrrgs); \\
    \quad{}\only<3->{\color{myred}}explore\_sea(me.love[0]);\only<3->{\color{black}}\\
    \quad{}shiver(me.timbers);             \\
    %\vspace*{-0.5\baselineskip}%
    \quad{}...
    }}
  \end{multicols}%

  \only<4->{
    \vspace{-0.5\baselineskip}
    Correct merge depends on the version in the merge base:\\
    {\footnotesize\texttt{%
      \quad{}speak\_like\_a\_pirate(arrrgs);\\
      \quad{}%
      %\only<4|handout:0>{{\color{myblue}?????}}%
      \only<4-8|handout:0>{{\color{myblue}?????}}%
      %\only<5-6|handout:0>{{\color{myblue}explore\_sea(aye, matey);}}%
      %\only<7-8|handout:0>{{\color{myblue}explore\_sea(me.love[0]);}}%
      \only<9->{{\color{myblue}explore\_sea(plus, plus);}}%
      \\
      \quad{}shiver(me.timbers);\\
    }}
  }

  %\only<6-10|handout:1>{
  \only<10|handout:1>{
  \vspace*{0.5\baselineskip}
  Which results in the following merge:\\
  {\footnotesize\texttt{%
    \quad{}speak\_like\_a\_pirate(arrrgs);\\
    \quad{}%
    %\only<7,9|handout:0>{\qquad}%
    %\only<6|handout:0>{{\color{mygreen}explore\_sea(me.love[0]);}}%
    %\only<8|handout:0>{{\color{mygreen}explore\_sea(aye, matey);}}%
    \only<10>{{\color{mygreen}
    <{}<{}<{}<{}<{}<{}< HEAD  \\
    \quad{}explore\_sea(aye, matey); \\
    \quad{}=======\\
    \quad{}explore\_sea(me.love[0]); \\
    \quad{}>{}>{}>{}>{}>{}>{}> branchB
    }}
    \\
    \quad{}shiver(me.timbers);\\
  }}
  }

  \only<11-| handout:2>{
    \vspace*{-0.5\baselineskip}
    \begin{multicols}{2}
      Shorthand:\\
      {\footnotesize\texttt{%
      \quad{}\phantom{Base1: }path\\
      \quad{}Base\phantom{1}: hash\_orig\\
      \quad{}Side1: hash\_A\\
      \quad{}Side2: hash\_B\\
      }}
    \columnbreak
    \uncover<12->{
      Example:\\
      {\footnotesize\texttt{%
      \quad{}\phantom{Base1: }buccaneer.c\\
      \quad{}Base\phantom{1}: ba771ed\\
      \quad{}Side1: 57abbed\\
      \quad{}Side2: b1a57ed
      }}
    }
    \end{multicols}%
  }

  \only<13| handout:2>{
    \vspace*{-0.5\baselineskip}
    {\footnotesize%
    
      Note: If any two of the hashes match, we can resolve without
      looking at the contents of the file.
      
    }
  }

\end{frame}

%%%%%%%%%%%%%%%%%%%%%%%%%%%%%%%%%%%%%%%%%%%%%%%%%%%%%%%%%%%%%%%%%%%%%%%%%%
\subsection[Basic merging]{Combining content merges}
%%%%%%%%%%%%%%%%%%%%%%%%%%%%%%%%%%%%%%%%%%%%%%%%%%%%%%%%%%%%%%%%%%%%%%%%%%

%%%%%%%%%%%%%%%%%%%%%%%%%%%%%%%%%%%%%%%%%%%%%%%%%%%%%%%%%%%%%%%%%%%%%%%%%%

\begin{frame}
  \frametitle{Three-way Merging}

  \cl{git checkout master}\\
  \cl{git merge feature}\\[0.5em]

  \uncover<3->{
  Get three relevant trees}\uncover<4->{, then for each path:
  \begin{itemize}
    \item Get version of path in each tree
    \item Do three-way content merge
  \end{itemize}
  }

  \vspace*{-1.5em}

  \pause
  \begin{center}
    \begin{minipage}{0.5\textwidth}
    \begin{block}{Merge}
      \begin{tikzpicture}[node distance=10mm]
        \only<2->{
          \node[commit] (A) {A};
        }
        \only<2,6>{
          \node[commit] (B)  [right of=A] {B};
          \node[commit] (D)  [right of=B] {D};
          \node[commit] (C)  [below right of=A,yshift=-1em] {C};
        }
        \only<3-5>{
          \node[orphanCommit] (B)  [right of=A] {B};
          \node[orphanCommit] (D)  [right of=B] {D};
          \node[orphanCommit] (C)  [below right of=A,yshift=-1em] {C};
        }
        \only<2->{
          \node[commit] (E)  [right of=D] {E};
          \path[prnt] (E) edge (D);
          \path[prnt] (D) edge (B);
          \path[prnt] (B) edge (A);

          \node[commit] (F)  [right of=C] {F};
          \path[prnt] (F) edge (C);
          \path[prnt] (C) edge (A);

          \node[ref]    (f) [below of=F] {feature};
          \path[bptr] (f) edge (F);
          \node[ref]    (m)  [above of=E] {master};
        }

        \only<2-5>{
          \path[bptr] (m) edge (E);
        }
        \only<5>{
          \node[redCommit] (G)  [right of=E] {G};
        }
        \only<6| handout:0>{
          \node[commit] (G)  [right of=E] {G};
        }
        \only<5->{
          \path[prnt] (G) edge (E);
          \path[prnt] (G) edge[bend left] (F);
        }
        \only<6| handout:0>{
          \path[bptr] (m) edge (G);
        }
        \only<0| handout:1>{
          \path[bptrnext] (m) edge (G);
        }
      \end{tikzpicture}
    \end{block}
    \end{minipage}
  \end{center}

\end{frame}

%%%%%%%%%%%%%%%%%%%%%%%%%%%%%%%%%%%%%%%%%%%%%%%%%%%%%%%%%%%%%%%%%%%%%%%%%%

\begin{frame}[fragile]
  \frametitle{Three-way Merging}

  \cl{git checkout master}\\
  \cl{git cherry-pick C..feature}\\[0.5em]

  \vspace*{-1.5em}
  \begin{center}
    \begin{minipage}{0.5\textwidth}
    \begin{block}{
        \only<1  | handout:0>{Starting state}
        \only<2-4| handout:1>{Picking F}
        \only<5  | handout:0>{Picked F}
        \only<6-8| handout:2>{Picking G}
        \only<9  | handout:0>{Picked G}
        \only<10-| handout:0>{Cherry picking complete}
        \phantom{Cg}% To make sure all titles have the same height
        }
      \begin{tikzpicture}[node distance=10mm]
        \only<1->{
          \node[commit] (A) {A};
          \node[commit] (B)  [right of=A] {B};
          \node[commit] (D)  [right of=B] {D};
          \node[commit] (E)  [right of=D] {E};
          \node[commit] (C)  [below right of=A,yshift=-1em] {C};
          \node[commit] (F)  [right of=C] {F};
          \node[commit] (G)  [right of=F] {G};
        }
        \only<3-7|handout:1>{
          \node[orphanCommit] (A)  {A};
          \node[orphanCommit] (B)  [right of=A] {B};
          \node[orphanCommit] (D)  [right of=B] {D};
          \node[orphanCommit] (G)  [right of=F] {G};
        }
        \only<1->{
          \path[prnt] (E) edge (D);
          \path[prnt] (D) edge (B);
          \path[prnt] (B) edge (A);

          \path[prnt] (G) edge (F);
          \path[prnt] (F) edge (C);
          \path[prnt] (C) edge (A);

          \node[ref]  (f) [below of=G] {feature};
          \path[bptr] (f) edge (G);
          \node[ref]  (m) [above of=E] {master};
        }

        \only<1-9| handout:1-2>{
          \path[bptr] (m) edge (E);
          %\node[ref]    (h)  [left of=m,xshift=-1.8em] {HEAD};
          %\path[bptr] (h) edge (m);
        }
        \only<4| handout:1>{
          \node[redCommit] (FS) [right of=E, xshift=0.1em] {F$^*$};
          \path[oprnt] (FS) edge[bend left] (F);
        }
        \only<4-| handout:1-2>{
          \path[prnt] (FS) edge (E);
        }
        \only<0| handout:1>{
          \path[bptrnext] (m) edge[bend left] (FS);
        }
        \only<5-| handout:2>{
          \node[commit] (A) {A};
          \node[commit] (B)  [right of=A] {B};
          \node[commit] (D)  [right of=B] {D};
          \node[commit] (FS) [right of=E, xshift=0.1em] {F$^*$};
          \node[commit] (F)  [right of=C] {F};
          \node[commit] (G)  [right of=F] {G};
        }
        \only<7-8| handout:2>{
          \node[orphanCommit] (A) {A};
          \node[orphanCommit] (B)  [right of=A] {B};
          \node[orphanCommit] (D)  [right of=B] {D};
          \node[orphanCommit] (E)  [right of=D] {E};
          \node[orphanCommit] (C)  [below right of=A,yshift=-1em] {C};
        }
        \only<8| handout:2>{
          \node[redCommit] (GS) [right of=FS, xshift=0.3em] {G$^*$};
          \path[oprnt] (GS) edge[bend left] (G);
        }
        \only<9-| handout:0>{
          \node[commit] (GS) [right of=FS, xshift=0.3em] {G$^*$};
        }
        \only<8-| handout:2>{
          \path[prnt] (GS) edge (FS);
        }
        \only<10-| handout:0>{
          \path[bptr] (m) edge[bend left] (GS);
          %\node[ref]    (h)  [left of=m,xshift=-1.8em] {HEAD};
          %\path[bptr] (h) edge (m);
        }
        \only<0| handout:2>{
          \path[bptrnext] (m) edge[bend left] (GS);
        }
      \end{tikzpicture}
    \end{block}
    \end{minipage}
  \end{center}

  \only<11>{
  Rebasing and reverting are handled similarly to cherry-picking.
  }

\end{frame}

%%%%%%%%%%%%%%%%%%%%%%%%%%%%%%%%%%%%%%%%%%%%%%%%%%%%%%%%%%%%%%%%%%%%%%%%%%
\subsection[Renames]{Why renames are important}
%%%%%%%%%%%%%%%%%%%%%%%%%%%%%%%%%%%%%%%%%%%%%%%%%%%%%%%%%%%%%%%%%%%%%%%%%%

\begin{frame}
  \frametitle{Why renames are important}

  \vspace*{-\baselineskip}
  {\scriptsize
  \begin{multicols}{2}
    \begin{minipage}{0.45\textwidth}
    \begin{block}{If a rename is not detected:}
        {\texttt{%
        \begin{tabular}{lll}
                 & buccaneer.c  & viking.c  \\
          Base:  & ba771e5      & 0000000 \\
          Side1: & e5ca185      & 0000000 \\
          Side2: & 0000000      & defea75 \\
        \end{tabular}
        }}

      \uncover<2->{
        \vspace*{1em}
        Then:
        \begin{itemize}[<+(1)->]
          \item buccaneer.c: modify/delete conflict
          \item viking.c: totally new file
          \item no textual merging
        \end{itemize}
      }

      \uncover<5->{
        \vspace*{1em}
        As reported by git status:\\[0.5em]
        {\texttt{%
          \quad{}Changes to be committed:\\
          \quad{}\enskip{}      new file: viking.c\\
          \quad{}Unmerged paths:\\
          \quad{}\enskip{}	deleted by them: buccaneer.c\\
        }}
      }
    \end{block}
    \end{minipage}

  \columnbreak

    \only<6->{
    \begin{minipage}{0.48\textwidth}
    \begin{block}{If we detect renames on each side of history:}
        \texttt{%
        \begin{tabular}{ll}
                  & buccaneer.c $\Rightarrow$ viking.c   \\
          Base:   & ba771e5 \\
          Side1:  & e5ca185 \\
          Side2:  & defea75 \\
          \uncover<7->{%
          Merged: & acc0575
          }
        \end{tabular}
        }

      \uncover<8->{
        \vspace*{1em}
        Then:
        \begin{itemize}[<+(4)->]
          \item buccaneer.c: removed
          \item viking.c: contains merged content
        \end{itemize}
      }

      \uncover<10->{
        \vspace*{1em}
        As reported by git status:\\[0.5em]
        EITHER\\
        {\texttt{%
          \quad{}Changes to be committed:\\
          \quad{}\,	renamed: buccaneer.c\hspace*{-0.25em}  ->\hspace*{-0.25em} viking.c\\
        }}
        OR\\
        {\texttt{%
          \quad{}Changes to be committed:\\
          \quad{}\,	deleted: buccaneer.c \\
          \quad{}Unmerged paths:\\
          \quad{}\,\,	both modified:   viking.c\\
        }}
      }

%      \begin{comment}
%      \only<11->{
%        \vspace*{0.6em}
%        ...and do a three-way content merge:\\[0.25em]
%        {\texttt{%
%          \quad{}speak\_like\_a\_pirate(arrrgs);\\
%          {\color{mygreen}
%          \quad{}<{}<{}<{}<{}<{}<{}< HEAD{\color{myred}:buccaneer.c}  \\
%          \quad{}explore\_sea(aye, matey); \\
%          \quad{}=======\\
%          \quad{}explore\_sea(me.love[0]); \\
%          \quad{}>{}>{}>{}>{}>{}>{}> branchB{\color{myred}:viking.c}\\
%          }
%          \quad{}shiver(me.timbers);\\
%        }}
%       }
%      \end{comment}

    \end{block}
    \end{minipage}
    } % end of \only<6->
  \end{multicols}
  } % end of \scriptsize

\end{frame}

%%%%%%%%%%%%%%%%%%%%%%%%%%%%%%%%%%%%%%%%%%%%%%%%%%%%%%%%%%%%%%%%%%%%%%%%%%
\subsection[Detection]{How rename detection works}
%%%%%%%%%%%%%%%%%%%%%%%%%%%%%%%%%%%%%%%%%%%%%%%%%%%%%%%%%%%%%%%%%%%%%%%%%%

\begin{frame}
  \frametitle{How rename detection works}

  How does git detect renames?  For each side...\\[0.25em]
  \pause
  \vspace*{-1.0\baselineskip}
  \begin{center}
    {\footnotesize
    \begin{tabular}{l|l}
      Files in Base  &  Files in given side \\
      \hline
      README.md      &  README.md   \\
      archery.js     &  corrupt.js  \\
      baseball.js    &  divine.js   \\
      build.log      &  dull.js     \\
      football.js    &  grand.js    \\
      golf.js        &  lame.js     \\
      running.js     &
    \end{tabular}
    }
  \end{center}
  \vspace*{-1.5\baselineskip}

  \pause
  \vspace*{\baselineskip}
  For each pair of files, what percentage of lines are found in both?
  \pause
  \vspace*{-0.25\baselineskip}
  \begin{center}
    {\footnotesize
    \begin{tabular}{l|l|l|l|l|l}
                   & corrupt.js & divine.js & dull.js & grand.js & lame.js \\
      \hline
      archery.js   &&&&& \\
      baseball.js  &&&&& \\
      build.log    &&&&& \\
      football.js  &&&&& \\
      golf.js      &&&&& \\
      running.js   &&&&&
    \end{tabular}
    }
  \end{center}
  \uncover<5| handout:1>{
    \begin{tikzpicture}[remember picture, overlay]
      \node[draw,ellipse,rotate=15,color=mygreen] at (7.2, 1.5)
           {Matrix of similarity percentages};
    \end{tikzpicture}
  }

\end{frame}

%%%%%%%%%%%%%%%%%%%%%%%%%%%%%%%%%%%%%%%%%%%%%%%%%%%%%%%%%%%%%%%%%%%%%%%%%%

\begin{frame}
  \frametitle{How rename detection works}

    \vfill

    \begin{center}
    \begin{minipage}{0.83\textwidth}
    \begin{block}{\begin{center}Crux of the problem\end{center}}
      \begin{center}
        \vspace*{1\baselineskip}

        Rename detection is O($M*N$), where $M$ and $N$
        are \textbf{huge}.

        \vspace*{2\baselineskip}

        {\scriptsize
          \{{\color{mygreen}$M$}, {\color{orange}$N$}\}
          \textasciitilde{} O(combined line count of potential rename
          \{{\color{mygreen}sources}, {\color{orange}targets}\})
        }
      \end{center}
    \end{block}
    \end{minipage}
    \end{center}

    \vfill

\end{frame}

%%%%%%%%%%%%%%%%%%%%%%%%%%%%%%%%%%%%%%%%%%%%%%%%%%%%%%%%%%%%%%%%%%%%%%%%%%
\section[Strategies]{Strategies to improve rename performance}
%%%%%%%%%%%%%%%%%%%%%%%%%%%%%%%%%%%%%%%%%%%%%%%%%%%%%%%%%%%%%%%%%%%%%%%%%%

\subsection[Bypassing]{Exact renames}

\begin{frame}
  \frametitle{}

  \vfill
  {\Large
  \begin{center}
    Optimization 1: Don't redo work\\
    \vspace*{\baselineskip}
    \pause
    Don't look for a better than perfect match.
  \end{center}
  }
  \vfill
\end{frame}

%%%%%%%%%%%%%%%%%%%%%%%%%%%%%%%%%%%%%%%%%%%%%%%%%%%%%%%%%%%%%%%%%%%%%%%%%%

\begin{frame}[fragile]
  \frametitle{Exact renames}

  \vspace*{-1.5\baselineskip}
  \begin{center}
  \begin{minipage}{0.83\textwidth}
  \begin{block}{Detecting renames}
    {\scriptsize
    \begin{semiverbatim}
void detect_renames_and_copies(...)
\{
    ...\uncover<2->{{\color{blue}
    exact_count = find_different_name_same_hash();   }\uncover<3->{{\color{myred}
    \only<1-3>{/* Keep all the source files as options for copies! */}   }}}
    for (dest_path in potential_rename_targets) \{   \uncover<2->{{\color{blue}
        if (already_paired(dest_path)) continue;   }}
        for (source_path in potential_rename_sources) \{   \uncover<3->{\uncover<4->{{\color{mygreen}
            if (!DETECT_COPIES &&
                already_paired(source_path))
                continue;   }}}
            compute_similarity();
        \}
    \}
    ...
\}  \end{semiverbatim}
    }
  \end{block}
  \end{minipage}
  \end{center}

  %\only<5->{
  %  \vspace*{-0.25\baselineskip}
  %  No sense wasting time trying to find a better match than a
  %  100\% rename.
  %}

  %\only<6>{
  %  \vspace*{-0.5\baselineskip}
  %  Speed-up: varies; my first testcase saw a factor of 2.
  %}

\end{frame}

%%%%%%%%%%%%%%%%%%%%%%%%%%%%%%%%%%%%%%%%%%%%%%%%%%%%%%%%%%%%%%%%%%%%%%%%%%

\subsection[Capitulating]{Partial capitulation}
\begin{frame}
  \frametitle{}

  \vfill
  {\Large
  \begin{center}
    Optimization 2: Don't do unnecessary work\\
    \vspace*{\baselineskip}
    \pause
    If you can get the same answer without an expensive computation,
    skip the expensive computation.
  \end{center}
  }
  \vfill
\end{frame}

%%%%%%%%%%%%%%%%%%%%%%%%%%%%%%%%%%%%%%%%%%%%%%%%%%%%%%%%%%%%%%%%%%%%%%%%%%

\begin{frame}
  \frametitle{Partial capitulation}

  \vspace*{-\baselineskip}
  \only<2->{
  {\scriptsize
  \begin{multicols}{2}
    \begin{minipage}{0.48\textwidth}
    \begin{block}{If a rename is not detected for the merge:}
        {\texttt{%
        \begin{tabular}{lll}
                 & buccaneer.c  & viking.c \\
          Base:
            &
            \only<2  |handout:0>{5eac0a57}%
            \only<3- |handout:1-2>{{\color{myred}5eac0a57}}%
            &
            00000000  \\
          Side1:
            &
            \only<2  |handout:0>{5caff01d}%
            \only<3  |handout:0>{{\color{myred}5caff01d}}%
            \only<4- |handout:1-2>{{\color{mygreen}5eac0a57}}
            &
            00000000  \\
          Side2: & 00000000      &  c01055a1 \\
        \end{tabular}
        }}

        \vspace*{1em}
        Then:
        \begin{itemize}
          \item buccaneer.c:
                \only<2-4|handout:0>{modify/delete conflict}%
                \only<5-8|handout:1>{{\color{myred}modify/delete conflict}}%
                \only<9- |handout:2>{{\color{mygreen}deleted as expected}}
          \item viking.c: totally new file
          \item
                \only<2-4|handout:0>{no textual merging}%
                \only<5-9|handout:1>{\color{myred}no textual merging}%
                \only<10-|handout:2>{no textual merging \color{mygreen}needed}
        \end{itemize}

        \vspace*{1em}
        As reported by git status:\\[0.5em]
        {\texttt{%
          \only<2-4|handout:0>{
            \quad{}Changes to be committed:\\
            \quad{}\enskip{}      new file: viking.c\\
            \quad{}Unmerged paths:\\
            \quad{}\enskip{}	deleted by them: buccaneer.c\\
          }%
          \only<5-10|handout:1>{{\color{myred}
            \quad{}Changes to be committed:\\
            \quad{}\enskip{}      new file: viking.c\\
            \quad{}Unmerged paths:\\
            \quad{}\enskip{}	deleted by them: buccaneer.c\\
          }}%
          \only<11-|handout:2>{{\color{mygreen}
            \quad{}Changes to be committed:\\
            \quad{}\,	renamed: buccaneer.c\hspace*{-0.25em}  ->\hspace*{-0.25em} viking.c\\
            \vspace*{2\baselineskip}
          }}
        }}
    \end{block}
    \end{minipage}

  \columnbreak

    \begin{minipage}{0.48\textwidth}
    \begin{block}{If we detect renames on each side of history:}
        \texttt{%
        \begin{tabular}{ll}
                 & buccaneer.c $\Rightarrow$ viking.c \\
          Base:
            &
            \only<2  |handout:0>{5eac0a57}%
            \only<3- |handout:1-2>{{\color{myred}5eac0a57}} \\
          Side1:
            &
            \only<2  |handout:0>{5caff01d}%
            \only<3  |handout:1>{{\color{myred}5caff01d}}%
            \only<4- |handout:2>{{\color{mygreen}5eac0a57}} \\
          Side2: &  c01055a1 \\
          Merged:
            &
            \only<2-4|handout:0>{0b57ac1e}%
            \only<5-6|handout:1>{{\color{myred}????????}}%
            \only<7- |handout:2>{{\color{mygreen}c01055a1}}
        \end{tabular}
        }

        \vspace*{1em}
        Then:
        \begin{itemize}
          \item buccaneer.c: removed
          \item viking.c:
                  \only<1-4|handout:0>{contains merged content}%
                  \only<5  |handout:1>{{\color{myred}contains merged content}}%
                  \only<6- |handout:2>{{\color{mygreen}no content merge was needed}}
        \end{itemize}

        \vspace*{1em}
        As reported by git status:\\[0.5em]
        \only<2-4|handout:0>{
          EITHER\\
          {\texttt{%
            \quad{}Changes to be committed:\\
            \quad{}\,	renamed: buccaneer.c\hspace*{-0.25em}  ->\hspace*{-0.25em} viking.c\\
          }}
          OR\\
          {\texttt{%
            \quad{}Changes to be committed:\\
            \quad{}\,	deleted: buccaneer.c \\
            \quad{}Unmerged paths:\\
            \quad{}\,\,	both modified:   viking.c\\
          }}
        }
        \only<5-7|handout:1>{{\color{myred}
          EITHER\\
          {\texttt{%
            \quad{}Changes to be committed:\\
            \quad{}\,	renamed: buccaneer.c\hspace*{-0.25em}  ->\hspace*{-0.25em} viking.c\\
          }}
          OR\\
          {\texttt{%
            \quad{}Changes to be committed:\\
            \quad{}\,	deleted: buccaneer.c \\
            \quad{}Unmerged paths:\\
            \quad{}\,\,	both modified:   viking.c\\
          }}
        }}
        \only<8-|handout:2>{{\color{mygreen}
          {\texttt{%
            \quad{}Changes to be committed:\\
            \quad{}\,	renamed: buccaneer.c\hspace*{-0.25em}  ->\hspace*{-0.25em} viking.c\\
          }}
          \vspace*{6\baselineskip}
        }}

    \end{block}
    \end{minipage}
  \end{multicols}

  \only<12>{
    \vspace*{-3\baselineskip}
    \begin{minipage}{0.48\textwidth}
    \begin{center}
      Same results whether or not rename is detected by merge machinery.
    \end{center}
    \end{minipage}
  }
  }} % end of \scriptsize


\end{frame}

%%%%%%%%%%%%%%%%%%%%%%%%%%%%%%%%%%%%%%%%%%%%%%%%%%%%%%%%%%%%%%%%%%%%%%%%%%

\begin{frame}[fragile]
  \frametitle{Partial capitulation -- Caveats?}

  \vspace*{-1.5\baselineskip}
  \begin{center}
  \begin{minipage}{0.8\textwidth}
  \begin{block}{New Strategy}
    Exclude potential source from rename detection \textbf{if} it is
    unmodified by \textit{other} side of history\only<1-3|handout:0>{.
      \vspace*{\baselineskip}
    }%
    \only<4-|handout:1>{
      {\color{mygreen}
      and parent directory of source file exists on \textit{same} side of
      history}.
    }

  \end{block}
  \end{minipage}
  \end{center}

  \only<2->{
  \vspace*{-0.5\baselineskip}
  Possible problems:
  \only<3->{
  \begin{itemize}
    \item
          \only<3-4|handout:0>{causes issues for directory rename detection}
          \only<5->{\redout{causes issues for directory rename detection}}
    \only<6->{
    \item rename/add conflict looks like add/add
    \only<7->{
    \item rename/rename(2to1) conflict looks like rename/add or add/add
    }}
  \end{itemize}
  }}

  \only<8->{
  ``Mis-detected'' conflict types:
  \only<9->{
  \begin{itemize}
    \item
      \only<9-12|handout:0>{Different conflict-related files in the working copy}
      \only<13->{\redout{Different conflict-related files in the working copy}}
    \only<10->{
    \item
      \only<10-12|handout:0>{Different conflict-related entries in the index}
      \only<13->{\redout{Different conflict-related entries in the index}}
    \only<11->{
    \item Different stdout; reports e.g. \texttt{CONFLICT(add/add)} instead
          of \texttt{CONFLICT(rename/add)}
    }}
  \end{itemize}
  }}

  \only<12->{
  After unifying file collision conflict handling...%
  \only<14>{{\color{mygreen}stdout is only difference.}}
  }

\end{frame}

%%%%%%%%%%%%%%%%%%%%%%%%%%%%%%%%%%%%%%%%%%%%%%%%%%%%%%%%%%%%%%%%%%%%%%%%%%

\begin{frame}[fragile]
  \frametitle{Partial capitulation -- micro or mega optimization?}

  \vspace*{-1.5\baselineskip}
  \begin{center}
  \begin{minipage}{0.8\textwidth}
  \begin{block}{New Strategy}
    Exclude potential source from rename detection \textbf{if} it is
    unmodified by \textit{other} side of history and parent directory
    of source file exists on \textit{same} side of history.
  \end{block}
  \end{minipage}
  \end{center}

  \only<2->{
    \begin{center}
    How much does this new strategy help?
    \end{center}
  }
  \only<3->{
    \vspace*{-1.0\baselineskip}
    \begin{center}
    \begin{minipage}{0.4\textwidth}
    \begin{block}{\begin{center}A Common Case\end{center}}
      \qquad
      O($M*N$) $\to$ \only<4>{O($\emptyset*N$)}\phantom{$\emptyset$}
    \end{block}
    \end{minipage}
    \end{center}
  }

\end{frame}

%%%%%%%%%%%%%%%%%%%%%%%%%%%%%%%%%%%%%%%%%%%%%%%%%%%%%%%%%%%%%%%%%%%%%%%%%%

\subsection[Linearizing]{Dimensionality Reduction}
\begin{frame}
  \frametitle{}

  \vfill
  {\Large
  \begin{center}
    Optimization 3: Fudge ``unnecessary''\\
    \vspace*{\baselineskip}
    \pause
    Only do part of the work and accept slightly different results if
    there are huge cost savings.
  \end{center}
  }
  \vfill
\end{frame}

%%%%%%%%%%%%%%%%%%%%%%%%%%%%%%%%%%%%%%%%%%%%%%%%%%%%%%%%%%%%%%%%%%%%%%%%%%

\begin{frame}
  \frametitle{Dimensionality Reduction}

  %Basic rename detection rules:
  %\only<2->{
  %\begin{itemize}
  %  \item If the same path exists in base and a side, those paths are
  %        always assumed to be different versions of the same file.  (No
  %        break detection is done.)
  %  \only<3->{
  %  \item If a path exists in base but not the side or vice versa, must
  %        exhaustively look through all pairings for best match.
  %  }
  %\end{itemize}
  %}

  %\only<4->{
    %\vspace*{-1.0\baselineskip}
    \vfill
    \begin{center}
    \begin{minipage}{0.65\textwidth}
    \begin{block}{\begin{center}Fun fact\end{center}}
        Over 75\% of renames in the linux kernel repository do not change
        the basename of the file, just the directory in which it is found.
    \end{block}
    \end{minipage}
    \end{center}
    \vfill
  %}

\end{frame}

%%%%%%%%%%%%%%%%%%%%%%%%%%%%%%%%%%%%%%%%%%%%%%%%%%%%%%%%%%%%%%%%%%%%%%%%%%

\begin{frame}[fragile]
  \frametitle{Dimensionality Reduction}

  \vspace*{-0.5\baselineskip}
  Detecting renames...

  \vspace*{-0.5\baselineskip}
  \begin{center}
    {\footnotesize
    \begin{tabular}{l|l}
      Files in Base  &  Files in given side \\
      \hline
        \only<1|handout:0>{\phantom{document.html}}%
        \only<2->{document.html}
      &
        \only<2->{build.log}
      \\

        \only<2->{src/{\color{blue}        blue.css}}
      &
        \only<2->{document.html}
      \\

        \only<2->{src/{\color{mybrown}     brown.css}}
      &
        \only<2->{source/{\color{blue}     blue.css}}
      \\

        \only<2->{src/{\color{mygreen}     green.css}}
      &
        \only<2->{source/{\color{mybrown}  brown.css}}
      \\

        \only<2->{src/{\color{myred}       red.css}}
      &
        \only<2->{source/{\color{mygreen}  green.css}}
      \\

      &
        \only<1|handout:0>{\phantom{source/{\color{orange}   orange.css}}}%
        \only<2->{source/{\color{orange}   orange.css}}
      \\

      &
        \only<2->{source/{\color{mypurple} purple.css}}
      \\

      &
        \only<2->{source/{\color{myred}    red.css}}

    \end{tabular}
    }
  \end{center}
  \vspace*{-1.5\baselineskip}

  \vspace*{\baselineskip}
  For each pair of files, what percentage of lines are found in both?
  \vspace*{-0.25\baselineskip}
  \begin{center}
    {\footnotesize
    \begin{tabular}{l|l|l|l|l}
          &
            \only<1-2|handout:0>{\phantom{src/blue.css}}
            \only<3->{src/{\color{blue}     blue.css}}
          &
            \only<1-2|handout:0>{\phantom{src/brown.css}}%
            \only<3->{src/{\color{mybrown}  brown.css}}
          &
            \only<1-2|handout:0>{\phantom{src/green.css}}%
            \only<3->{src/{\color{mygreen}  green.css}}
          &
            \only<1-2|handout:0>{\phantom{src/red.css}}%
            \only<3->{src/{\color{myred}    red.css}}
          \\

      \hline

      \only<3->{build.log}           &&&& \\
      \only<3->{source/{\color{blue}     blue.css}}    &&&& \\
      \only<3->{source/{\color{mybrown}  brown.css}}   &&&& \\
      \only<3->{source/{\color{mygreen}  green.css}}   &&&& \\
      \only<1-2|handout:0>{\phantom{source/orange.css}}%
      \only<3->{source/{\color{orange}   orange.css}}  &&&& \\
      \only<3->{source/{\color{mypurple} purple.css}}  &&&& \\
      \only<3->{source/{\color{myred}    red.css}}     &&&& \\


    \end{tabular}
    }
  \end{center}
  \only<4| handout:1>{
    \begin{tikzpicture}[remember picture, overlay]
      \node[draw,ellipse,rotate=10,color=cyan] at (7.7, 1.6)
           {Matrix of similarity percentages};
    \end{tikzpicture}
  }
  \only<5| handout:1>{
    \begin{tikzpicture}[remember picture, overlay]
      \node[draw,ellipse,rotate=10,color=cyan] at (7.7, 1.7)
           {Sparse similarity percentages};
    \end{tikzpicture}
  }

\end{frame}

%%%%%%%%%%%%%%%%%%%%%%%%%%%%%%%%%%%%%%%%%%%%%%%%%%%%%%%%%%%%%%%%%%%%%%%%%%

\begin{frame}
  \frametitle{Dimensionality Reduction}

  %Basic rename detection rules:
  %\begin{itemize}
  %  \item If the same path exists in base and a side, those paths are
  %        always assumed to be different versions of the same file.  (No
  %        break detection is done.)
  %  \only<2->{{\color{mygreen}
  %  \item If unique basename matches are found, compare the contents.
  %        If their similarity exceeds the rename threshold treat them
  %        as a rename without looking for a ``better''
  %        match.  Otherwise...
  %  }}
  %  \item If a path exists in base but not the side or vice versa, must
  %        exhaustively look through all pairings for best match.
  %\end{itemize}

  %\only<3->{

    \vspace*{3\baselineskip}
    \begin{center}
    \begin{minipage}{0.5\textwidth}
    \begin{block}{\begin{center}Improvement\end{center}}
        \quad
        O($M*N$) $\to$ O((M-B)*(N-B) \only<1>{\phantom{+ B}}\only<2->{+ B})
    \end{block}
    \end{minipage}
    \end{center}

    \only<3>{
    \vspace*{1\baselineskip}
    \begin{center}
    \begin{minipage}{0.5\textwidth}
    \begin{block}{\begin{center}If enough matching basenames...\end{center}}
        \quad
        O($M*N$) $\to$ O(minimum($M,N$))
    \end{block}
    \end{minipage}
    \end{center}
    }
  %}

\end{frame}

%%%%%%%%%%%%%%%%%%%%%%%%%%%%%%%%%%%%%%%%%%%%%%%%%%%%%%%%%%%%%%%%%%%%%%%%%%

\subsection[Remembering]{Remembering previous work}
\begin{frame}
  \frametitle{}

  \vfill
  {\Large
  \begin{center}
    Optimization 4: Don't redo unnecessary work\\
    \vspace*{\baselineskip}
    \pause
    When repeatedly merging, re-use\\ previous rename detection results.
  \end{center}
  }
  \vfill
\end{frame}

%%%%%%%%%%%%%%%%%%%%%%%%%%%%%%%%%%%%%%%%%%%%%%%%%%%%%%%%%%%%%%%%%%%%%%%%%%

\begin{frame}
  \frametitle{Remembering previous work}

  \qquad
  \begin{minipage}{0.5\textwidth}
    \begin{block}{
        \only<1   |handout:0>{Cherry-picking C..feature}
        \only<2-6 |handout:1>{Picking F}
        \only<7   |handout:0>{Picked F}
        \only<8-15|handout:2>{Picking G}
        %\only<16  |handout:0>{Picked G}
        %\only<17- |handout:0>{Cherry picking complete}
        \phantom{Cg}% To make sure all titles have the same height
        }
      \begin{tikzpicture}[node distance=10mm]
        \only<1->{
          \node[commit] (A) {A};
          \node[commit] (B)  [right of=A] {B};
          \node[commit] (D)  [right of=B] {D};
          \node[commit] (E)  [right of=D] {E};
          \node[commit] (C)  [below right of=A,yshift=-1em] {C};
          \node[commit] (F)  [right of=C] {F};
          \node[commit] (G)  [right of=F] {G};
        }
        \only<3-9|handout:2>{
          \node[orphanCommit] (A)  {A};
          \node[orphanCommit] (B)  [right of=A] {B};
          \node[orphanCommit] (D)  [right of=B] {D};
          \node[orphanCommit] (G)  [right of=F] {G};
        }
        \only<1->{
          \path[prnt] (E) edge (D);
          \path[prnt] (D) edge (B);
          \path[prnt] (B) edge (A);

          \path[prnt] (G) edge (F);
          \path[prnt] (F) edge (C);
          \path[prnt] (C) edge (A);

          \node[ref]  (f) [below of=G] {feature};
          \path[bptr] (f) edge (G);
          \node[ref]  (m) [above of=E] {master};
        }

        %\only<1-16| handout:1-2>{
        \only<1-15| handout:1-2>{
          \path[bptr] (m) edge (E);
          %\node[ref]    (h)  [left of=m,xshift=-1.8em] {HEAD};
          %\path[bptr] (h) edge (m);
        }
        \only<6|handout:1>{
          \node[redCommit] (FS) [right of=E, xshift=0.1em] {F$^*$};
          \path[oprnt] (FS) edge[bend left] (F);
        }
        \only<6->{
          \path[prnt] (FS) edge (E);
        }
        \only<7-| handout:2>{
          \node[commit] (A) {A};
          \node[commit] (B)  [right of=A] {B};
          \node[commit] (D)  [right of=B] {D};
          \node[commit] (FS) [right of=E, xshift=0.1em] {F$^*$};
          \node[commit] (F)  [right of=C] {F};
          \node[commit] (G)  [right of=F] {G};
        }
        \only<9-15| handout:2>{
          \node[orphanCommit] (A) {A};
          \node[orphanCommit] (B)  [right of=A] {B};
          \node[orphanCommit] (D)  [right of=B] {D};
          \node[orphanCommit] (E)  [right of=D] {E};
          \node[orphanCommit] (C)  [below right of=A,yshift=-1em] {C};
        }
        \only<15|handout:2>{
          \node[redCommit] (GS) [right of=FS, xshift=0.3em] {G$^*$};
          \path[oprnt] (GS) edge[bend left] (G);
        }
        %\only<16-|handout:0>{
        %  \node[commit] (GS) [right of=FS, xshift=0.3em] {G$^*$};
        %}
        \only<15-| handout:2>{
          \path[prnt] (GS) edge (FS);
        }
        %\only<17-| handout:0>{
        %  \path[bptr] (m) edge[bend left] (GS);
        %  %\node[ref]    (h)  [left of=m,xshift=-1.8em] {HEAD};
        %  %\path[bptr] (h) edge (m);
        %}
      \end{tikzpicture}
    \end{block}
  \end{minipage}%
  \only<7-|handout:2>{%
  \hfill
  \begin{minipage}{0.32\textwidth}
    \begin{block}{}
      $
      \begin{array}{ll}
        \multicolumn{2}{c}{\textrm{buccaneer.c}\Rightarrow\textrm{viking.c}} \\
        C: & \texttt{c0a575} \\
        E: & \texttt{b1ade5} \\
        F: & \texttt{befa11} \\
        \hline
        F^*\!\!: & \texttt{1007ed}
      \end{array}
      $
    \end{block}
  \end{minipage}
  \hfill
  }

  \vspace*{-\baselineskip}
  \only<4-7|handout:1>{
    \vfill
    \begin{center}
    \begin{minipage}{0.88\textwidth}
    \begin{block}{}
      $
      \left\lbrace
      \begin{array}{lll}
           & \textrm{buccaneer.c} & \textrm{viking.c} \\
        C: & \texttt{c0a575}      & \texttt{000000}   \\
        E: & \texttt{000000}      & \texttt{b1ade5}   \\
        F: & \texttt{befa11}      & \texttt{000000}
      \end{array}
      \right\rbrace \only<5->{\to \left\lbrace
      \begin{array}{ll}
        \multicolumn{2}{c}{\textrm{buccaneer.c}\Rightarrow\textrm{viking.c}} \\
        C: & \texttt{c0a575} \\
        E: & \texttt{b1ade5} \\
        F: & \texttt{befa11}
      \end{array}\right\rbrace
      }
      $

      \only<4-5|handout:0>{
        \phantom{$F^*$}
      }
      \only<6-7>{
        \hspace*{16.4em} $F^*:$ \;\;\,\texttt{1007ed}
      }
    \end{block}
    \end{minipage}
    \end{center}
    \vfill
  }
  \only<10-|handout:2>{
    \vfill
    \begin{center}
    \begin{minipage}{0.88\textwidth}
    \begin{block}{}
      $
      \left\lbrace
      \begin{array}{lll}
             & \textrm{buccaneer.c} & \textrm{viking.c} \\
        F:   & \texttt{befa11}      & \texttt{000000}   \\
        F^*: & \texttt{000000}      & \texttt{1007ed}   \\
        G:   & \texttt{a70115}      & \texttt{000000}
      \end{array}
      \right\rbrace \only<13->{\to \left\lbrace
      \begin{array}{ll}
        \multicolumn{2}{c}{\textrm{buccaneer.c}\Rightarrow\textrm{viking.c}} \\
        F:   & \texttt{befa11}  \\
        F^*: & \texttt{1007ed}  \\
        G:   & \texttt{a70115}
      \end{array}\right\rbrace
      }$

      \only<10-13|handout:0>{
        \phantom{$G^*$}
      }
      \only<14->{
      \hspace*{17.0em} $G^*:$ \;\;\,\texttt{fab1ed}
      }
    \end{block}
    \end{minipage}
    \end{center}
    \vfill
  }
  \only<11-13|handout:2>{
    \begin{tikzpicture}[remember picture,overlay]
       \draw[mygreen,line width=1pt] (2.93, 2.11) ellipse (2.5em and 0.5em);
       \draw[orange, line width=1pt] (5.06, 1.64) ellipse (2.5em and 0.5em);
       \only<12-13>{
       \draw[mygreen,line width=1pt] (9.39, 4.97) ellipse (2.5em and 0.5em);
       \draw[orange, line width=1pt] (9.39, 4.47) ellipse (2.5em and 0.5em);
       }
    \end{tikzpicture}
  }

\end{frame}

%%%%%%%%%%%%%%%%%%%%%%%%%%%%%%%%%%%%%%%%%%%%%%%%%%%%%%%%%%%%%%%%%%%%%%%%%%

\subsection[Honing]{But wait, there's more!}
\begin{frame}
  \frametitle{But wait, there's more!}

  \begin{itemize}

    \item Avoid accidentally quadratic behavior

    \item Restructure to eliminate quasi-quadratic index insertion and
          removal

    \item Fewer tree traversals

    \item Extend ``partial capitulation'' ideas from \textit{file}
          renames to \textit{directory} renames

      \begin{itemize}
        \item More caveats involved, but...
        \item Also enables partial tree traversal elimination
      \end{itemize}

    \item Avoid updating the index or working tree if not needed

      \begin{itemize}
        \item Helps with new \only<1>{sparse-checkout}\only<2->{{\color{mygreen}sparse-checkout}} command
        \item Accelerates rebases and cherry-picks
        \item \only<1-2>{Avoids unnecessary recompilation}%
              \only<3>{{\color{mygreen}Avoids unnecessary recompilation}} after a rebase
      \end{itemize}

    \item ...and a few other minor improvements
      % * avoid re-comparing hashes (unpack_trees did it once, then we repeated)
      % * memoize not-found results from previous runs of check_dir_renamed()
      %   (store dir_renames + non-renames in a strmap)

  \end{itemize}

\end{frame}

%%%%%%%%%%%%%%%%%%%%%%%%%%%%%%%%%%%%%%%%%%%%%%%%%%%%%%%%%%%%%%%%%%%%%%%%%%
\section[Results]{Results}
%%%%%%%%%%%%%%%%%%%%%%%%%%%%%%%%%%%%%%%%%%%%%%%%%%%%%%%%%%%%%%%%%%%%%%%%%%

\begin{frame}
  \frametitle{Results}

  %Linux kernel, backport hwmon-updates from 5.5 to 5.4-ish.
  \vfill
  \begin{center}
    %In Progress\\[2em]

    %https://github.com/newren/git/tree/ort

    \includegraphics[scale=0.5]{timing.pdf}
  \end{center}
  \vfill

\end{frame}

%%%%%%%%%%%%%%%%%%%%%%%%%%%%%%%%%%%%%%%%%%%%%%%%%%%%%%%%%%%%%%%%%%%%%%%%%%

\begin{frame}
  \frametitle{The journey, redux}

  \vspace*{-0.5\baselineskip}
  \begin{minipage}{0.88\textwidth}
  \begin{block}{Issues starting the journey}
    {\footnotesize
    \begin{itemize}
      \item cherry-pick would fail to detect renames and fail to notify
            about needed merge.renameLimit
      \item cherry-pick would ignore merge.renameLimit > 32767
      \item if directory renames involved, files would be left in wrong
            directory
      \item people wrote custom purpose scripts to cherry-pick things
      \item after fixing merge.renameLimit, cherry-picking small patches
            would take more than 9 minutes.
    \end{itemize}
    }
  \end{block}
  \end{minipage}

  \vspace*{\baselineskip}
  {\small
  \only<2->{
    When I told folks a few years ago that "You don't need these
    special scripts to cherry pick things; just set merge.renameLimit
    to something higher," they responded that merge.renameLimit didn't
    work.\\[1em]

    \only<3->{I didn't believe them.\\[1em]}

    \only<4>{My efforts in this area, including this performance work,
      represent my attempt to continue to not believe them.  :-)}
  }}

\end{frame}

%%%%%%%%%%%%%%%%%%%%%%%%%%%%%%%%%%%%%%%%%%%%%%%%%%%%%%%%%%%%%%%%%%%%%%%%%%

\end{document}
