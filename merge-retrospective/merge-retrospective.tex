\ifx\handout\undefined
  \documentclass[compress,t]{beamer}
\else
  \documentclass[compress,handout,t]{beamer}
  \usepackage{pgfpages}
  %\pgfpagesuselayout{8 on 1}[letterpaper,border shrink=1mm]
  \pgfpagesuselayout{8 on 1}[letterpaper,border shrink=0mm]
\fi

\usepackage{amsmath}                  % Nice math stuff
\usepackage{amsfonts}                 % Extra math fonts (i.e., \mathbb{})
\usepackage{amssymb}                  % Extra meth symbols
\usepackage[mathscr]{eucal}           % Scripted math fonts
\usepackage{listings}                 % Formatting code listings
\usepackage{verbatim}
%\usepackage{picins}
\usepackage{colortbl}
\usepackage{fancybox}
\usepackage{textpos}
\usepackage{color}

\mode<presentation>
{
  \usetheme{Warsaw} % or ...
  \setbeamercovered{transparent} % or whatever (possibly just delete it)
  \usefonttheme[onlymath]{serif}
}
\mode<beamer>
{
  \AtBeginSection[]{
    \begin{frame}
      \frametitle{Outline}
      \begin{textblock}{0.9}(0.1,0.0)
        {\small\tableofcontents[currentsection,hideothersubsections]}
      \end{textblock}
    \end{frame}
  }
}
% PKN: the footline isn't very useful -- better to conserve pixels
\setbeamertemplate{footline}{}
\setbeamertemplate{navigation symbols}{}
\setbeamersize{text margin left=0.2cm}
\setbeamersize{text margin right=0.2cm}
\setbeamersize{sidebar width left=0cm}
\setbeamersize{sidebar width right=0cm}
%\setbeamertemplate{footline}
%{% 
%\begin{beamercolorbox}{section in head/foot} 
%\vskip2pt\insertnavigation{\paperwidth}\vskip2pt 
%\vskip2pt
%\hfill Page \insertpagenumber \hspace{1em}
%\end{beamercolorbox}% 
%} 

\usepackage[english]{babel}           % or whatever
\usepackage[latin1]{inputenc}         % or whatever
\usepackage{times}
\usepackage[T1]{fontenc}

\newcommand{\Comment}[1]{{\usebeamercolor[fg]{item}#1}}
\newcommand{\Text}[1]{\texttt{#1}}
\xdefinecolor{TokenColor}{rgb}{0.000, 0.700, 0.000}
\newcommand{\Token}[1]{{\color{TokenColor}#1}}
\xdefinecolor{HiddenColor}{rgb}{0.800, 0.800, 0.800}
\newcommand{\Hidden}[1]{{\color{HiddenColor}#1}}
\xdefinecolor{MyRed}{rgb}{0.800, 0.200, 0.200}
\newcommand{\Red}[1]{{\color{MyRed}#1}}
\xdefinecolor{MyWhite}{rgb}{1.0, 1.0, 1.0}
\xdefinecolor{MyGrey}{rgb}{0.900, 0.900, 0.900}

\setlength{\TPHorizModule}{\textwidth}
\setlength{\TPVertModule}{\textheight}

\lstloadlanguages{C++}
\lstset{%
  basicstyle=\ttfamily\scriptsize\bfseries, %
  commentstyle=\usebeamercolor[fg]{item}, %
  stringstyle=\color{TokenColor}, %
  language=C++}

%% Typeset an example command line input.
%\newcommand{\cmd}[1]{\textbf{\texttt{#1}}}
%\newcommand{\cl}[1]{\$ \textbf{#1}}
%\newcommand{\cls}[1]{\texttt{\scriptsize\$ \textbf{#1}}}
%\newcommand{\clsnp}[1]{\texttt{\scriptsize\textbf{#1}}}
\newcommand{\hilight}{\color{blue}}
\xdefinecolor{MyGreen}{rgb}{0.000, 0.500, 0.000} % green
\newcommand{\nonliteral}[1]{{\color{MyGreen}#1}}
\newcommand{\cmd}[1]{\texttt{\hilight#1}}
\newcommand{\cl}[1]{\$ \cmd{#1}}

%% typset a URL in a uniform way
\newcommand{\Link}[1]{{\usebeamercolor[fg]{item}\underline{\url{#1}}}}

\usepackage{multicol}
\usepackage{ulem} % used in definition of \redout

\usepackage{tikz}
\usetikzlibrary{arrows,snakes,backgrounds,shapes}
\tikzstyle{commit}=[circle,draw=blue!50,fill=blue!20]
\tikzstyle{redCommit}=[circle,draw=MyRed!50,fill=MyRed!20]
\tikzstyle{redConflict}=[circle,draw=MyRed!50,dashed,fill=MyRed!20]
\tikzstyle{orphanCommit}=[black!50,circle,draw=blue!20,fill=blue!10]
\tikzstyle{droppedCommit}=[black!60,circle,draw=MyRed!50,fill=MyRed!20,double]
\tikzstyle{ref}=[rectangle,draw=black!50,fill=black!20]
\tikzstyle{specialRef}=[black!50,rectangle,draw=black!50,fill=black!10]
\tikzstyle{msg}=[tape,tape bend top=none,draw=MyRed!50,fill=MyRed!20]
\tikzstyle{prnt}=[thick,-stealth']
\tikzstyle{oprnt}=[black!50,dotted,-stealth',draw=black!50]
\tikzstyle{bptrnext}=[black!80,dashed,-stealth',draw=black!80]
\tikzstyle{bptr}=[thick,-latex']
\tikzstyle{nextmsg}=[thick,-open triangle 45]
\newcommand{\parent}[1]{\char94{#1}}
\newcommand{\ancestor}[1]{\char126{#1}}
\colorlet{myred}{red!90!black} % slightly darker than normal red
\colorlet{myblue}{cyan}        % plain cyan works; might consdier darkening
\colorlet{mygreen}{green!70!black} % slightly darker than normal green
\colorlet{mypurple}{red!45!blue}
\definecolor{mybrown}{rgb}{0.45,0.28,0.13}
\newcommand\redout{\bgroup\markoverwith
    {\textcolor{myred}{\rule[.5ex]{2pt}{0.8pt}}}\ULon
}

%\title{(Git Merge)$²$}
\title{Git Merge}

\author{Elijah Newren}
\institute{Palantir Technologies}

%\date{September 14, 2022}
%\date{Palantir Technologies}
\date{}

%\includeonlyframes{wip}

\begin{document}

%%%%%%%%%%%%%%%%%%%%%%%%%%%%%%%%%%%%%%%%%%%%%%%%%%%%%%%%%%%%%%%%%%%%%%%%%%

\begin{frame}
  \titlepage
\end{frame}

\ifx\handout\undefined
\title{(Git Merge)$^2$}
\begin{frame}
  \titlepage
\end{frame}

\title{New Merge Backend -- Why, How, What}
\begin{frame}
  \titlepage
\end{frame}
\fi

%%%%%%%%%%%%%%%%%%%%%%%%%%%%%%%%%%%%%%%%%%%%%%%%%%%%%%%%%%%%%%%%%%%%%%%%%%
\section[Background]{New Merge Backend background}
\subsection{Scope}
%%%%%%%%%%%%%%%%%%%%%%%%%%%%%%%%%%%%%%%%%%%%%%%%%%%%%%%%%%%%%%%%%%%%%%%%%%

\begin{frame}
  \frametitle{Scope}

  The merge machinery powers several aspects of git:

  \begin{itemize}
    \item merge
    \item cherry-pick
    \item revert
    \item rebase
    \item stash
    \item switch -m, checkout -m
    \item am -3
  \end{itemize}


\end{frame}

\begin{comment}
%%%%%%%%%%%%%%%%%%%%%%%%%%%%%%%%%%%%%%%%%%%%%%%%%%%%%%%%%%%%%%%%%%%%%%%%%%
\subsection{New strategy}
%%%%%%%%%%%%%%%%%%%%%%%%%%%%%%%%%%%%%%%%%%%%%%%%%%%%%%%%%%%%%%%%%%%%%%%%%%

\begin{frame}
  \frametitle{New strategy or algorithm}

  \only<1>{
  Git has the concept of pluggable strategies or algorithms; built-in ones
  include:
  \begin{itemize}
    \item recursive (prior default)
    \item resolve
    \item octopus
    \item subtree
    \item ours
  \end{itemize}
  }

  \only<2->{
  These can be selected at run-time:\\
  \quad\cl{git merge -s <strategy> $\ldots$}
  }

  \only<3->{
  \vspace*{0.5\baselineskip}
  For example:\\
  \quad\cl{git merge -s recursive $\ldots$}
  }

  \only<4->{
  \vspace*{2\baselineskip}

  I wrote a replacement for the default strategy, recursive.  Since
  the parser didn't require a space:\\
  \quad\cl{git merge -s<strategy> $\ldots$}
  }

  \only<5->{
  \vspace*{2\baselineskip}
  I called the new strategy ``\texttt{ort}'':\\
  \quad\cl{git merge -sort $\ldots$}
  }

\end{frame}
\end{comment}

%%%%%%%%%%%%%%%%%%%%%%%%%%%%%%%%%%%%%%%%%%%%%%%%%%%%%%%%%%%%%%%%%%%%%%%%%%
\subsection{Three-way merge}
%%%%%%%%%%%%%%%%%%%%%%%%%%%%%%%%%%%%%%%%%%%%%%%%%%%%%%%%%%%%%%%%%%%%%%%%%%

\begin{frame}
  \frametitle{Three-way content merge}

  \begin{center}
  \vfill
  \begin{minipage}{0.6\textwidth}
  \begin{center}
  Basic idea behind merge:\\
  Do three-way content merge on each file.
  \end{center}
  \end{minipage}

  \vfill
  \pause
  \begin{minipage}{0.6\textwidth}
  \begin{center}
  Purpose of three-way content merge:\\
  Reconcile differences when both sides
  of history made changes.
  \end{center}
  \end{minipage}

  \vfill
  \end{center}

\end{frame}

%%%%%%%%%%%%%%%%%%%%%%%%%%%%%%%%%%%%%%%%%%%%%%%%%%%%%%%%%%%%%%%%%%%%%%%%%%

\begin{frame}
  \frametitle{Three-way content merge}

  \vspace*{-\baselineskip}
  \begin{multicols}{2}
    File from branch Side1:\\
    {\footnotesize\texttt{%
    %\vspace*{-0.125\baselineskip}%
    \quad{}...                             \\
    \quad{}speak\_like\_a\_pirate(arrrgs); \\
    \quad{}\only<3->{\color{myred}}explore\_sea(aye, matey);\only<3->{\color{black}}\\
    \quad{}shiver(me.timbers);             \\
    %\vspace*{-0.25\baselineskip}%
    \quad{}...                             \\
    }}
    \columnbreak
    \pause
    Same file from branch Side2:\\
    {\footnotesize\texttt{%
    %\vspace*{-0.125\baselineskip}%
    \quad{}...                             \\
    \quad{}speak\_like\_a\_pirate(arrrgs); \\
    \quad{}\only<3->{\color{myred}}explore\_sea(me.love[0]);\only<3->{\color{black}}\\
    \quad{}shiver(me.timbers);             \\
    %\vspace*{-0.5\baselineskip}%
    \quad{}...
    }}
  \end{multicols}%

  \only<4->{
    \vspace{-0.5\baselineskip}
    Correct merge depends on the version in the merge base:\\
    {\footnotesize\texttt{%
      \quad{}speak\_like\_a\_pirate(arrrgs);\\
      \quad{}%
      \only<4|handout:0>{{\color{myblue}?????}}%
      \only<5|handout:0>{{\color{myblue}explore\_sea(aye, matey);}}%
      \only<6|handout:0>{{\color{myblue}explore\_sea(me.love[0]);}}%
      \only<7->{{\color{myblue}explore\_sea(plus, plus);}}%
      \\
      \quad{}shiver(me.timbers);\\
    }}
  }

  %\only<6-10|handout:1>{
  \only<5-|handout:1>{
  \begin{center}
  \begin{minipage}{0.5\textwidth}
  Which results in the following merge:\\
  {\footnotesize\texttt{%
    \quad{}speak\_like\_a\_pirate(arrrgs);\\
    \quad{}%
    \only<5|handout:0>{{\color{mygreen}explore\_sea(me.love[0]);}}%
    \only<6|handout:0>{{\color{mygreen}explore\_sea(aye, matey);}}%
    \only<7>{{\color{mygreen}
    <{}<{}<{}<{}<{}<{}< Side1  \\
    \quad{}explore\_sea(aye, matey); \\
    \quad{}=======\\
    \quad{}explore\_sea(me.love[0]); \\
    \quad{}>{}>{}>{}>{}>{}>{}> Side2
    }}
    \\
    \quad{}shiver(me.timbers);\\
  }}
  \end{minipage}
  \end{center}
  }

\end{frame}

%%%%%%%%%%%%%%%%%%%%%%%%%%%%%%%%%%%%%%%%%%%%%%%%%%%%%%%%%%%%%%%%%%%%%%%%%%
\section{Advantages}
\subsection{{-}{-}remerge-diff}
%%%%%%%%%%%%%%%%%%%%%%%%%%%%%%%%%%%%%%%%%%%%%%%%%%%%%%%%%%%%%%%%%%%%%%%%%%

\begin{frame}
  \frametitle{Advantage: New {-}{-}remerge-diff feature}

  Sometimes you want to investigate the conflict resolution performed
  for a merge, including whether unrelated changes were included:\\
  \vspace*{0.5\baselineskip}
  \qquad\cl{git show {-}{-}remerge-diff me\_commit}\\
  or\\
  \qquad\cl{git log {-}{-}remerge-diff me\_branch}\

  \only<2>{
  {\scriptsize
  \vspace*{\baselineskip}
    \quad{}diff {-}{-}git a/filename b/filename \\
    \quad{}remerge CONFLICT (content): Merge conflict in filename \\
    \quad{}index ba771ed..57abbed 100644 \\
    \quad{}{-}{-}{-} a/filename \\
    \quad{}{-}{-}{-} b/filename \\
    \quad{}@@ -106,7 +106,3 @@ \\
    \quad{}speak\_like\_a\_pirate(arrrgs); \\
  {\color{myred}
    \quad{}-<{}<{}<{}<{}<{}<{}< Side1  \\
    \quad{}-explore\_sea(aye, matey); \\
    \quad{}-=======\\
    \quad{}-explore\_sea(me.love[0]); \\
    \quad{}->{}>{}>{}>{}>{}>{}> Side2 \\
  }{\color{mygreen}
    \quad{}+explore\_sea(aye, tis, me.love[0]); \\
  }
    \quad{}shiver(me.timbers);
  }}

\end{frame}

%%%%%%%%%%%%%%%%%%%%%%%%%%%%%%%%%%%%%%%%%%%%%%%%%%%%%%%%%%%%%%%%%%%%%%%%%%
\subsection{AUTO\_MERGE}
%%%%%%%%%%%%%%%%%%%%%%%%%%%%%%%%%%%%%%%%%%%%%%%%%%%%%%%%%%%%%%%%%%%%%%%%%%

\begin{frame}[fragile]
  \frametitle{Advantage: New AUTO\_MERGE feature}

  Sometimes you want to investigate the conflict resolution you have
  done so far for a merge, including any unrelated changes:\\
  \vspace*{\baselineskip}
  \qquad\cl{git diff AUTO\_MERGE}

  \only<2>{
  {\scriptsize
  \vspace*{4.1\baselineskip}
    \quad{}diff {-}{-}git a/filename b/filename \\
    \quad{}index ba771ed..57abbed 100644 \\
    \quad{}{-}{-}{-} a/filename \\
    \quad{}{-}{-}{-} b/filename \\
    \quad{}@@ -106,7 +106,3 @@ \\
    \quad{}speak\_like\_a\_pirate(arrrgs); \\
  {\color{myred}
    \quad{}-<{}<{}<{}<{}<{}<{}< Side1  \\
    \quad{}-explore\_sea(aye, matey); \\
    \quad{}-=======\\
    \quad{}-explore\_sea(me.love[0]); \\
    \quad{}->{}>{}>{}>{}>{}>{}> Side2 \\
  }{\color{mygreen}
    \quad{}+explore\_sea(aye, tis, me.love[0]); \\
  }
    \quad{}shiver(me.timbers);
  }}

  \begin{comment}
  ALTERNATE:

  There are a variety of questions users might ask while resolving
  conflicts:
  \vspace*{-\baselineskip}
  \begin{enumerate}
    \item What changes have been made since the previous (first) parent?
    \item What changes are staged?
    \item What is still unstaged? (or what is still conflicted?)
    \item What changes did I make to resolve conflicts so far?
  \end{enumerate}

  \vspace*{\baselineskip}
  The first three of these have simple answers:
  \begin{enumerate}
    \item \texttt{git diff HEAD}
    \item \texttt{git diff {-}{-}cached}
    \item \texttt{git diff}
  \end{enumerate}

  Due to the new merge strategy, the fourth now has an answer too:
  \begin{enumerate}[4]
    \item \texttt{git diff AUTO\_MERGE}
  \end{enumerate}
  \end{comment}

\end{frame}

%%%%%%%%%%%%%%%%%%%%%%%%%%%%%%%%%%%%%%%%%%%%%%%%%%%%%%%%%%%%%%%%%%%%%%%%%%
\subsection[subsetting]{Repository subsets}
%%%%%%%%%%%%%%%%%%%%%%%%%%%%%%%%%%%%%%%%%%%%%%%%%%%%%%%%%%%%%%%%%%%%%%%%%%

\begin{frame}
  \frametitle{Advantages: working with repository subsets}

  The new merge backend is vital to the efforts for working with
  repository subsets in Git.
  \begin{itemize}
    \item sparse-checkout: avoid needless vivifying
    \item sparse-index: ability to work without full index
    \item partial clone: massive reduction in blobs needed
  \end{itemize}

\end{frame}

%%%%%%%%%%%%%%%%%%%%%%%%%%%%%%%%%%%%%%%%%%%%%%%%%%%%%%%%%%%%%%%%%%%%%%%%%%
\subsection[...]{More features coming}
%%%%%%%%%%%%%%%%%%%%%%%%%%%%%%%%%%%%%%%%%%%%%%%%%%%%%%%%%%%%%%%%%%%%%%%%%%

\begin{frame}
  \frametitle{Advantages: More features in the pipeline}

  \begin{itemize}
    \item Server side operations
    \item Merging/Rebasing/etc. changes disconnected from HEAD
    \item Replaying commits with less worktree dirtying
    \item Replaying merge commits and their ``hidden'' changes
  \end{itemize}

\end{frame}

%%%%%%%%%%%%%%%%%%%%%%%%%%%%%%%%%%%%%%%%%%%%%%%%%%%%%%%%%%%%%%%%%%%%%%%%%%
\subsection[Fixes]{Correctness}
%%%%%%%%%%%%%%%%%%%%%%%%%%%%%%%%%%%%%%%%%%%%%%%%%%%%%%%%%%%%%%%%%%%%%%%%%%

\begin{frame}
  \frametitle{Advantage: Correctness}

  Correctness:
  \begin{itemize}
    \item 34 tests that succeed with the new backend, but fail with the old one
    \item \hspace*{0.5em}0 tests that fail with the new backend, but succeed with the old one
  \end{itemize}

\end{frame}

%%%%%%%%%%%%%%%%%%%%%%%%%%%%%%%%%%%%%%%%%%%%%%%%%%%%%%%%%%%%%%%%%%%%%%%%%%
\subsection[Perf]{Performance}
%%%%%%%%%%%%%%%%%%%%%%%%%%%%%%%%%%%%%%%%%%%%%%%%%%%%%%%%%%%%%%%%%%%%%%%%%%

\begin{frame}
  \frametitle{Advantage: Performance}

  Performance:\\
  \vspace*{0.5\baselineskip}
  \only<1>{
  \qquad
  \begin{tabular}{l|r|c}%{l|r|c}
    Testcase       & \multicolumn{1}{c|}{Old timing}
                     %\multicolumn{1}{p{4.57em}}{Old timing \phantom{foobar}} \\
                   & \multicolumn{1}{p{4.93em}}{New timing \phantom{foobar}} \\
    \hline
    few-renames:   &   18.934 s & \\%& .1892 s \\
    mega-renames:  & 6424.660 s & \\%& .6573 s \\
    just-one-mega: &  173.093 s & \\%& .2612 s
  \end{tabular}
  }
  \only<2->{
  \qquad
  \begin{tabular}{l|r|c|c}
    Testcase       & \multicolumn{1}{c|}{Old timing}
                   & \multicolumn{1}{c|}{New timing}
                   & \multicolumn{1}{p{4.4em}}{\centering Speed-up factor} \\
    \hline
    few-renames:   &   18.934 s &
                   %& .1892 s
                   %&  100.1 \hspace*{0.3em} \\
                   & x \\
    mega-renames:  & 6424.660 s &
                   %& .6573 s
                   %& 9774.3 \hspace*{0.3em} \\
                   & y \\
    just-one-mega: &  173.093 s &
                   %& .2612 s
                   %&  662.7 \hspace*{0.3em}
                   & z
  \end{tabular}
  }%
  \only<3->{
  \vspace*{\baselineskip}
  \begin{center}Average speedup factor (geometric mean):
         %865.4
         $(x*y*z)^\frac{1}{3}$
  \end{center}
  }

\end{frame}

%%%%%%%%%%%%%%%%%%%%%%%%%%%%%%%%%%%%%%%%%%%%%%%%%%%%%%%%%%%%%%%%%%%%%%%%%%
\section[Discovery]{Discovering Optimizations}
\subsection[]{Background}
%%%%%%%%%%%%%%%%%%%%%%%%%%%%%%%%%%%%%%%%%%%%%%%%%%%%%%%%%%%%%%%%%%%%%%%%%%

\begin{frame}
  \frametitle{Why renames are important}

  If a rename from file A $\to$ file B is not detected:
  \begin{itemize}
    \item A is marked as modified on one side, deleted on the other (conflict!)
    \item B is marked as a new file
  \end{itemize}

  \only<2->{
  \vspace*{\baselineskip}
  This poses a problem:
  \begin{itemize}
    \item There is no hint that users should merge these files
    \item It's non-obvious how to properly merge these files
  \end{itemize}
  }

  \only<3->{
  \vspace*{\baselineskip}
  In my experience, users often handle this via one of:
  \begin{itemize}[<+(2)->]
    \item Just keeping both files (oops) %(each of which is missing changes made
          %on the other side of history)
    \item Just picking one of the two files (oops) %(unknowingly discarding
          %changes made on the other side of history)
    \item If users \textit{do} notice there was a rename, they
          manually hand apply the subset of changes they remember from
          one file to the other (oops) %(unknowingly discarding the
          %remaining subset of changes)
  \end{itemize}
  }

\end{frame}

%%%%%%%%%%%%%%%%%%%%%%%%%%%%%%%%%%%%%%%%%%%%%%%%%%%%%%%%%%%%%%%%%%%%%%%%%%

\begin{frame}
  \frametitle{How rename detection works}

  How does git detect renames?  For each side...\\[0.25em]
  \pause
  \vspace*{-0.5\baselineskip}
  \begin{center}
    {\footnotesize
    \begin{tabular}{l|l}
      Files in Base  &  Files in given side \\
      \hline
      README.md      &  README.md   \\
      archery.js     &  corrupt.js  \\
      baseball.js    &  divine.js   \\
      build.log      &  dull.js     \\
      football.js    &  grand.js    \\
      golf.js        &  lame.js     \\
      running.js     &
    \end{tabular}
    }
  \end{center}
  \vspace*{-0.5\baselineskip}

  \pause
  \vspace*{\baselineskip}
  For each deleted/added pair, what percentage of lines are found in both?
  \pause
  \vspace*{-0.25\baselineskip}
  \begin{center}
    {\footnotesize
    \begin{tabular}{l|l|l|l|l|l}
                   & corrupt.js & divine.js & dull.js & grand.js & lame.js \\
      \hline
      archery.js   &&&&& \\
      baseball.js  &&&&& \\
      build.log    &&&&& \\
      football.js  &&&&& \\
      golf.js      &&&&& \\
      running.js   &&&&&
    \end{tabular}
    }
  \end{center}
  \uncover<5| handout:1>{
    \begin{tikzpicture}[remember picture, overlay]
      \node[draw,ellipse,rotate=15,color=mygreen] at (7.2, 1.5)
           {Matrix of similarity percentages};
    \end{tikzpicture}
  }

\end{frame}

%%%%%%%%%%%%%%%%%%%%%%%%%%%%%%%%%%%%%%%%%%%%%%%%%%%%%%%%%%%%%%%%%%%%%%%%%%
\subsection[]{The Journey Begins}
%%%%%%%%%%%%%%%%%%%%%%%%%%%%%%%%%%%%%%%%%%%%%%%%%%%%%%%%%%%%%%%%%%%%%%%%%%

\begin{frame}
  \frametitle{How it all began}

  In a repository with tens of thousands of renames (due to dozens of
  high-ish level directory renames):

  \pause
  {\textcolor{red}{
  \begin{center}
    people wrote custom scripts to simulate cherry-picking a commit
  \end{center}
  }}

  \only<3->{
  \vspace*{\baselineskip}
  Because:
  \begin{itemize}[<+(2)->]
    \item There were silent caps on \texttt{merge.renameLimit} (since fixed)
    \item \texttt{git cherry-pick} with small patches took more than 9 minutes
  \end{itemize}
  }

\end{frame}

%%%%%%%%%%%%%%%%%%%%%%%%%%%%%%%%%%%%%%%%%%%%%%%%%%%%%%%%%%%%%%%%%%%%%%%%%%

\begin{frame}
  \vfill
  \vspace*{2\baselineskip}
  {\Huge
  \begin{center}Oversimplifications coming...\end{center}
  }
  \vfill
\end{frame}

%%%%%%%%%%%%%%%%%%%%%%%%%%%%%%%%%%%%%%%%%%%%%%%%%%%%%%%%%%%%%%%%%%%%%%%%%%
\subsection[Relevance]{Relevant Renames}
%%%%%%%%%%%%%%%%%%%%%%%%%%%%%%%%%%%%%%%%%%%%%%%%%%%%%%%%%%%%%%%%%%%%%%%%%%

\begin{frame}
  \vfill
  \vspace*{2\baselineskip}
  {\Huge
  \begin{center}Relevant renames\end{center}
  }
  \vfill
\end{frame}

%%%%%%%%%%%%%%%%%%%%%%%%%%%%%%%%%%%%%%%%%%%%%%%%%%%%%%%%%%%%%%%%%%%%%%%%%%

\begin{frame}
  \frametitle{Those custom ``cherry-pick'' scripts}

  Users wrote scripts that would:

  \begin{itemize}[<+(1)->]
    \item do a \texttt{git format-patch -1 \textit{\$COMMIT}}
    \item modify the file names according to a set of patterns
    \item run \texttt{git apply}
  \end{itemize}

  \only<5->{
  \vspace*{\baselineskip}
  The result?
  \only<6->{
  Successful cherry-pick in a matter of \textit{seconds}\only<6->{, usually}.
  }}

\end{frame}

%%%%%%%%%%%%%%%%%%%%%%%%%%%%%%%%%%%%%%%%%%%%%%%%%%%%%%%%%%%%%%%%%%%%%%%%%%

\begin{frame}
  \frametitle{Those custom ``cherry-pick'' scripts}

  Maddening question: Why can these scripts ignore tens of thousands
  of renames, and only pay attention to renames for the five files
  modified in the commit being ``cherry-picked''...\only<2->{and still usually
  get the right answer?}

  \only<3->{
  \begin{center}
  \vspace*{0.5\baselineskip}
  \begin{minipage}{0.6\textwidth}
  \begin{center}
  Purpose of rename detection in merges:
  \end{center}
  \vspace*{-0.5\baselineskip}
  \only<4->{
  Allow us to match up files, for three-way content merging\only<4-6>{.}%
  \only<4-6>{\vspace*{2\baselineskip}}
  \only<7->{\textcolor{red}{AND used to detect directory renames so we can
  move new files added to old directories to the new directory.}}
  }
  \end{minipage}

  \only<5->{
  \vspace*{1\baselineskip}
  \begin{minipage}{0.6\textwidth}
  \begin{center}
  If only one side modified the file, we don't need three-way content merges.
  \end{center}
  \end{minipage}
  }

  \only<6->{
  \vspace*{1\baselineskip}
  \begin{minipage}{0.6\textwidth}
  \begin{center}
  $\Rightarrow$ We can exclude \textit{most} source files from rename detection.
  \end{center}
  \end{minipage}
  }

  \end{center}
  }
  
\end{frame}

%%%%%%%%%%%%%%%%%%%%%%%%%%%%%%%%%%%%%%%%%%%%%%%%%%%%%%%%%%%%%%%%%%%%%%%%%%

\begin{frame}
  \frametitle{Relevant renames optimization}

  Only include \textit{relevant} source files in rename detection.

  \pause
  \vspace*{\baselineskip}
  \begin{tabular}{lr}
    Speed-up factor if only this optimization included:  & 75.8 \\
    %Slow-down factor if only this optimization excluded: & 4.9
  \end{tabular}

  \vspace*{2\baselineskip}
  \pause
  {\scriptsize
  Behavioral Change: Very Minor \& Battle Tested for over a Decade
  \pause
  \begin{itemize}
    \item If there are multiple valid rename pairings (multiple files that
          are very similar on each side of history), excluding some source
          files can cause other source files to be paired differently.
    \pause
    \item \texttt{git am -3} effectively embeds a naive form of this
          optimization through its algorithm of ``building fake
          ancestors''
          \pause
          (and is incompatible with directory rename detection) \\
          \pause
          \vspace*{\baselineskip}
          \only<6>{
          $\Rightarrow$ one inconsistency between rebase backends has been
          removed
          }\only<7>{
          $\Rightarrow$ \textcolor{mygreen}{one inconsistency between rebase
          backends has been removed
          }}
  \end{itemize}
  }

\end{frame}

%%%%%%%%%%%%%%%%%%%%%%%%%%%%%%%%%%%%%%%%%%%%%%%%%%%%%%%%%%%%%%%%%%%%%%%%%%
\subsection[No Copy]{No copy detection}
%%%%%%%%%%%%%%%%%%%%%%%%%%%%%%%%%%%%%%%%%%%%%%%%%%%%%%%%%%%%%%%%%%%%%%%%%%

\begin{frame}
  \vfill
  \vspace*{2\baselineskip}
  {\Huge
  \begin{center}No copy detection\end{center}
  }
  \vfill
\end{frame}

%%%%%%%%%%%%%%%%%%%%%%%%%%%%%%%%%%%%%%%%%%%%%%%%%%%%%%%%%%%%%%%%%%%%%%%%%%

\begin{frame}[fragile]
  \frametitle{Source code gems}

  Git stores hashes of individual files and trees (and commits), so:
  \pause
  \begin{itemize}
    \item If two files have the same hash, treat them as an exact rename and
          remove them from the similarity percentages matrix
  \end{itemize}

  \pause
  \vspace*{1.0\baselineskip}
  The implementation didn't quite match...\\
  \vspace*{0.5\baselineskip}
  \qquad
  \begin{minipage}{0.8\textwidth}
  {\footnotesize
  \begin{verbatim}
	rename_count = find_exact_renames(options);

	/*
	 * Calculate how many renames are left (but all the source
	 * files still remain as options for copies!)
	 */
	num_create = (rename_dst_nr - rename_count);
  \end{verbatim}
  }
  \end{minipage}

  \pause
  \vspace*{1.0\baselineskip}
  Merging has no use for copy detection and turns it off.
  \pause
  This code meant we wasted time looking for a \textit{better} match than
  an identical copy.

\end{frame}

%%%%%%%%%%%%%%%%%%%%%%%%%%%%%%%%%%%%%%%%%%%%%%%%%%%%%%%%%%%%%%%%%%%%%%%%%%

\begin{frame}
  \frametitle{Correctly exploiting exact renames optimization}

  Do not waste time looking for a \textit{better} match than an
  identical copy.

  \pause
  \vspace*{\baselineskip}
  \begin{tabular}{lr}
    Speed-up factor if only this optimization included:  & 2.83 \\
    %Slow-down factor if only this optimization excluded: & 1.02
  \end{tabular}

  \vspace*{2\baselineskip}
  \pause
  {\scriptsize
  Behavioral Change: None
  }

\end{frame}

%%%%%%%%%%%%%%%%%%%%%%%%%%%%%%%%%%%%%%%%%%%%%%%%%%%%%%%%%%%%%%%%%%%%%%%%%%
\subsection[Caching]{Caching renames}
%%%%%%%%%%%%%%%%%%%%%%%%%%%%%%%%%%%%%%%%%%%%%%%%%%%%%%%%%%%%%%%%%%%%%%%%%%

\begin{frame}
  \vfill
  \vspace*{2\baselineskip}
  {\Huge
  \begin{center}Caching renames\end{center}
  }
  \vfill
\end{frame}

%%%%%%%%%%%%%%%%%%%%%%%%%%%%%%%%%%%%%%%%%%%%%%%%%%%%%%%%%%%%%%%%%%%%%%%%%%

\begin{frame}
  \frametitle{Noticing patterns}

  When rebasing a long sequence of commits, before implementing other
  optimizations I noticed a pattern in the output:
  
  \vspace*{1.0\baselineskip}
  \qquad
  \begin{minipage}{0.8\textwidth}
    {\scriptsize
    Performing inexact rename detection: 100\% (32116945/32116945), done.\\
    Performing inexact rename detection: 100\% (32116945/32116945), done.\\
    Performing inexact rename detection: 100\% (32116945/32116945), done.\\
    Performing inexact rename detection: 100\% (32116945/32116945), done.\\
    \hspace*{0.4\textwidth}\vdots\\
    Performing inexact rename detection: 100\% (32116945/32116945), done.
    }
  \end{minipage}

  \pause
  \vspace*{1.0\baselineskip}
  Exact same numbers each time?  It's detecting the same 26 thousand
  renames, over and over!

  \pause
  \vspace*{1.0\baselineskip}
  Why not just remember them after the first time?

\end{frame}

%%%%%%%%%%%%%%%%%%%%%%%%%%%%%%%%%%%%%%%%%%%%%%%%%%%%%%%%%%%%%%%%%%%%%%%%%%

\begin{frame}
  \frametitle{Caching renames during a rebase or cherry-pick}

  Cache upstream renames during a rebase or cherry-pick, to reuse for
  later commits in the same operation.

  \pause
  \vspace*{\baselineskip}
  \begin{tabular}{lr}
    Speed-up factor if only this optimization included:  & 5.5 \\
    %Slow-down factor if only this optimization excluded: & 1.7
  \end{tabular}

  \vspace*{2\baselineskip}
  \pause
  {\scriptsize
  Behavioral Change: Extraordinarily Minor \& Always No Worse
  }

  \pause
  \vspace*{2\baselineskip}
  {\scriptsize
  Other notes:
  \begin{itemize}[<+(1)->]
    \item Simple idea, very complex to show it is safe;
          see \texttt{Documentation/technical/remembering-renames.txt} in
          the Git source.
    \item A more comprehensive cache or on disk record of renames would be
          either unsafe or actually slow things down.
  \end{itemize}
  }

\end{frame}

%%%%%%%%%%%%%%%%%%%%%%%%%%%%%%%%%%%%%%%%%%%%%%%%%%%%%%%%%%%%%%%%%%%%%%%%%%
\subsection[Basenames]{Basename-guided rename detection}
%%%%%%%%%%%%%%%%%%%%%%%%%%%%%%%%%%%%%%%%%%%%%%%%%%%%%%%%%%%%%%%%%%%%%%%%%%

\begin{frame}
  \vfill
  \vspace*{2\baselineskip}
  {\Huge
  \begin{center}Basename-guided rename detection\end{center}
  }
  \vfill
\end{frame}

%%%%%%%%%%%%%%%%%%%%%%%%%%%%%%%%%%%%%%%%%%%%%%%%%%%%%%%%%%%%%%%%%%%%%%%%%%

\begin{frame}
  \frametitle{Rename comparison matrix sizes}

  Rename detection involves comparing N source files with M target files,
  computing similarity percentages.  Matrix sizes:\\[1em]
  \pause
  \qquad
  \begin{tabular}{lc}
    At start:                     & 27328 x 26536 \\
    After exact rename detection: &  6077 x  5285 \\
    After relevance optimization: &   \hspace*{1.75em}1 x  5285
  \end{tabular}

  \only<3-6>{
  \vspace*{1.5\baselineskip}
  Can we identify likely rename pairs, without doing full similarity
  comparisons?
  \begin{center}
    {\footnotesize
    \only<3>{
      \begin{tabular}{l|l|l|l|l|l}
                     & corrupt.js & divine.js & dull.js & grand.js & lame.js \\
        \hline
        archery.js   &&&&& \\
        baseball.js  &&&&& \\
        build.log    &&&&& \\
        football.js  &&&&& \\
        golf.js      &&&&& \\
        running.js   &&&&&
      \end{tabular}
    }
    \only<4-5>{
      \begin{tabular}{l|l|l|l|l}
               & src/blue.css & src/brown.css & src/green.css & src/red.css \\
        \hline
        build.log         &&&& \\
        source/blue.css   &&&& \\
        source/brown.css  &&&& \\
        source/green.css  &&&& \\
        source/red.css    &&&& \\
        template.txt      &&&& \\
      \end{tabular}
    }
    \only<6>{
      \begin{tabular}{l|l|l|l|l}
                          & {src/\color{myblue}blue.css}
                          & {src/\color{orange}orange.css}
                          & {src/\color{mypurple}purple.css}
                          & {src/\color{myred}red.css} \\
        \hline
        build.log                           &&&& \\
        {source/\color{myblue}blue.css}     &&&& \\
        {source/\color{orange}orange.css}   &&&& \\
        {source/\color{mypurple}purple.css} &&&& \\
        {source/\color{myred}red.css}       &&&& \\
        template.txt                        &&&& \\
      \end{tabular}
    }
    }
  \end{center}
  \only<3-4>{
    \begin{tikzpicture}[remember picture, overlay]
      \node[draw,ellipse,rotate=15,color=mygreen] at (7.2, 1.5)
           {Matrix of similarity percentages};
    \end{tikzpicture}
  }
  }

  \only<7->{
  \vspace*{2\baselineskip}
  I have to
  compare \texttt{\textbf{\textit{olddir}}/{\color{mygreen}modified-filename.ext}}
  to 5285 files to find out it's a rename
  of \texttt{\textbf{\textit{newdir}}/{\color{mygreen}modified-filename.ext}}?

  \pause
  \vspace*{\baselineskip}
  Can I somehow use the fact that there is only
  one \texttt{\color{mygreen}modified-filename.ext} on each side of history?
  }
  
\end{frame}

%%%%%%%%%%%%%%%%%%%%%%%%%%%%%%%%%%%%%%%%%%%%%%%%%%%%%%%%%%%%%%%%%%%%%%%%%%

\begin{frame}
  \frametitle{Exploiting File Basenames}

  Percentage of historical renames that moved a file into a different
  directory without changing its basename:
  \pause
  \begin{itemize}
    \item linux: 76\%
    \item gcc: 64\%
    \item gecko: 79\%
    \item webkit: 89\%
  \end{itemize}

  \pause
  \vspace*{\baselineskip}
  Idea: Check if files with same basename are similar, with higher
  similarity threshold.  If met, remove that pair from matrix.

\end{frame}

\begin{frame}
  \frametitle{Basename-guided rename detection}

  Use file basenames to guide rename detection.

  \pause
  \vspace*{\baselineskip}
  \begin{tabular}{lr}
    Speed-up factor if only this optimization included:  & 5.8 \\
    %Slow-down factor if only this optimization excluded: & 2.0
  \end{tabular}

  \vspace*{2\baselineskip}
  \pause
  {\scriptsize
  Behavioral Change: Significant \& New
  \pause
  \begin{itemize}
    \item Could result in different file pairings if multiple possible matches
          (documented)
    \pause
    \item Controlled by (undocumented) \texttt{GIT\_BASENAME\_FACTOR}
          environment variable
    \pause
    \item Not unique to new merge strategy; we made it apply to the old
          strategy and to diffs as well.
  \end{itemize}
  }

\end{frame}

%%%%%%%%%%%%%%%%%%%%%%%%%%%%%%%%%%%%%%%%%%%%%%%%%%%%%%%%%%%%%%%%%%%%%%%%%%
\subsection[Directories]{Trivial directory resolution}
%%%%%%%%%%%%%%%%%%%%%%%%%%%%%%%%%%%%%%%%%%%%%%%%%%%%%%%%%%%%%%%%%%%%%%%%%%

\begin{frame}
  \vfill
  \vspace*{2\baselineskip}
  {\Huge
  \begin{center}Trivial directory resolution\end{center}
  }
  \vfill
\end{frame}

%%%%%%%%%%%%%%%%%%%%%%%%%%%%%%%%%%%%%%%%%%%%%%%%%%%%%%%%%%%%%%%%%%%%%%%%%%

\begin{frame}
  \frametitle{Three-way merge, revisited}

  Git tracks hashes of commits, of each subtree, and of each file.
  Three-way merge can thus be explained as starting at the toplevel
  trees for each of the three commits and:
  
  \begin{enumerate}[<+(1)->]
    \item Recursively walking into each tree, gathering all files
    \item Match up the files for three-way merges (using renames)
    \item Do three-way content merge of matched files
    \item Use the resulting files to write new subtrees and toplevel tree
  \end{enumerate}

  \pause
  \vspace*{\baselineskip}
  What if one of the subtrees matches between the merge base and one of the
  sides?

  \pause
  \vspace*{\baselineskip}
  Can we avoid recursing into it?

\end{frame}

%%%%%%%%%%%%%%%%%%%%%%%%%%%%%%%%%%%%%%%%%%%%%%%%%%%%%%%%%%%%%%%%%%%%%%%%%%

\begin{frame}
  \frametitle{Unlocking trivial directory resolution}

  Three key ideas:
  \begin{enumerate}[<+(1)->]
    \item If there are no renames, the optimization is safe.
    \item If there are no \textit{relevant} renames, the optimization is safe
    \item Defer recursing into potentially trivially resolvable directories
  \end{enumerate}

\end{frame}

%%%%%%%%%%%%%%%%%%%%%%%%%%%%%%%%%%%%%%%%%%%%%%%%%%%%%%%%%%%%%%%%%%%%%%%%%%

\begin{frame}
  \frametitle{Trivial directory resolution}

  Avoid recursing into subtrees where one side matches the merge base,
  where possible.

  \pause
  \vspace*{\baselineskip}
  \begin{tabular}{lr}
    Speed-up factor if only this optimization included[*]:  & 4.6 \\
    %Slow-down factor if only this optimization excluded: & 7.3
  \end{tabular}

  \vspace*{2\baselineskip}
  {\scriptsize
  \pause
  Behavioral Change: No New changes
  \begin{itemize}
    \item [*] Relies on the ``Relevant Renames'' optimization, which does bring
          along some changes (which were ``Very Minor \& Battle Tested'')
  \end{itemize}
  }

\end{frame}

%%%%%%%%%%%%%%%%%%%%%%%%%%%%%%%%%%%%%%%%%%%%%%%%%%%%%%%%%%%%%%%%%%%%%%%%%%
\section{Wrapping Up}
\subsection{Performance Summary}
%%%%%%%%%%%%%%%%%%%%%%%%%%%%%%%%%%%%%%%%%%%%%%%%%%%%%%%%%%%%%%%%%%%%%%%%%%

\begin{frame}
  \frametitle{Performance Summmary}

  \vspace*{-0.2\baselineskip}
  Overall results of individual optimizations:\\
  \pause
  \vspace*{0.4\baselineskip}
  \qquad
  \begin{tabular}{l|r}%{l|r|r}
    Optimization
      & \multicolumn{1}{p{5.5em}}{\centering Sole-source Speed-up}
      %& \multicolumn{1}{|p{5.5em}}{\centering Sole-source Slow-down}
      \\
    \hline
    Relevant renames                & 75.85\hspace*{1.4em}
                                    %& 4.90 \hspace*{1.4em}
                                    \\
    No copy detection               &  2.83\hspace*{1.4em}
                                    %& 1.02 \hspace*{1.4em}
                                    \\
    Caching renames                 &  5.52\hspace*{1.4em}
                                    %& 1.73 \hspace*{1.4em}
                                    \\
    Basename-guided detection       &  5.83\hspace*{1.4em}
                                    %& 2.01 \hspace*{1.4em}
                                    \\
    Trivial directory resolution[*] &  4.64\hspace*{1.4em}
                                    %& 7.27 \hspace*{1.4em}
  \end{tabular}

  \pause
  \vspace*{0.4\baselineskip}
  Overall optimization results, by testcase:\\
  \vspace*{0.4\baselineskip}
  \qquad
  \begin{tabular}{l|r|c|r}
    Testcase       & \multicolumn{1}{c|}{Old timing}
                   & \multicolumn{1}{c|}{New timing}
                   & \multicolumn{1}{p{4.4em}}{\centering Speed-up factor} \\
    \hline
    few-renames:   &   18.934 s & .1892 s &  100.1 \hspace*{0.3em} \\
    mega-renames:  & 6424.660 s & .6573 s & 9774.3 \hspace*{0.3em} \\
    just-one-mega: &  173.093 s & .2612 s &  662.7 \hspace*{0.3em}
  \end{tabular}\\
  \begin{center}Overall average speed-up: 865.4\end{center}

\end{frame}

%%%%%%%%%%%%%%%%%%%%%%%%%%%%%%%%%%%%%%%%%%%%%%%%%%%%%%%%%%%%%%%%%%%%%%%%%%
\subsection{Blog}
%%%%%%%%%%%%%%%%%%%%%%%%%%%%%%%%%%%%%%%%%%%%%%%%%%%%%%%%%%%%%%%%%%%%%%%%%%

\begin{frame}
  \frametitle{Blog Posts}

  Those curious about more details on the optimizations are encouraged
  to read my blog posts on the subject:

  \qquad
  \begin{center}
  Optimizing Git's Merge Machinery:\\ \href{https://blog.palantir.com/optimizing-gits-merge-machinery-1-127ceb0ef2a1}{\textcolor{blue}{Part I}}, 
  \href{https://blog.palantir.com/optimizing-gits-merge-machinery-2-d81391b97878}{\textcolor{blue}{part II}}, 
  \href{https://blog.palantir.com/optimizing-gits-merge-machinery-3-2dc7c7436978}{\textcolor{blue}{part III}}, 
  \href{https://blog.palantir.com/optimizing-gits-merge-machinery-part-iv-5bbc4703d050}{\textcolor{blue}{part IV}}, 
  \href{https://blog.palantir.com/optimizing-gits-merge-machinery-part-v-46ff3710633e}{\textcolor{blue}{part V}}, and
  \href{https://blog.palantir.com/optimizing-gits-merge-machinery-6-7bf887a131d8}{\textcolor{blue}{part VI}}.
  \end{center}

\end{frame}

%%%%%%%%%%%%%%%%%%%%%%%%%%%%%%%%%%%%%%%%%%%%%%%%%%%%%%%%%%%%%%%%%%%%%%%%%%

\subsection{}
\begin{frame}
  \vfill
  \vspace*{2\baselineskip}
  {\Huge
  \begin{center}Thanks\end{center}
  }
  \vfill
\end{frame}

%%%%%%%%%%%%%%%%%%%%%%%%%%%%%%%%%%%%%%%%%%%%%%%%%%%%%%%%%%%%%%%%%%%%%%%%%%

\end{document}
